\documentclass[a4paper,11pt,twoside]{article}
% Encoding lokal, da manche Editoren danach schauen
\usepackage[utf8]{inputenc} 
\usepackage{ihr-15}

\title{Entwurfsbeschreibung „Interaktiver Haushaltsplanrechner“}
 
\date{Version vom 9. Oktober 2015}

\begin{document}
\maketitle
\tableofcontents
\newpage
\seitezwei
\newpage

\section{Einführung}

Der im Rahmen des Projekts erstellte Prototyp eines neuen Interaktiven
Haushaltsplanrechners für die Stadt Leipzig verbindet drei Elemente modernen
Designs:
\begin{itemize}
\item Die Datenschicht setzt auf moderne semantische Konzepte auf RDF-Basis,
  was eine leichte Anbindung der Anwendung an die Linked Open Data Welt im
  Rahmen einer zukünftigen Open Data Strategie der Stadt Leipzig ermöglicht. 
\item Die Präsentationsschicht setzt mit Drupal und Visualisierungen auf der
  Basis des verbreiteten Javascript-Frameworks D3 auf nachhaltige
  Standardlösungen aus dem Open Source Bereich, die perspektivisch von einem
  kleineren, einschlägig qualifizierten IT-Team betreut und weiterentwickelt
  werden können.
\item Partizipative Elemente sind als „kleine Lösung“ über einen
  Drupal-Forums-Baustein in die Anwendung integriert, lassen sich aber in eine
  (noch zu entwickelnde) dezentrale, modular vernetzte „große Lösung“ eines
  anwendungsübergreifenden Partizipationskonzepts integrieren.
\end{itemize}
Dieses Konzept und dessen Realisierung wurde in drei Etappen entwickelt und
umgesetzt.

In einer \emph{ersten Etappe} (Oktober 2014 bis Januar 2015) analysierte ein
interdisziplinär zusammengesetztes studentisches Team Gestaltungsvarianten und
Erfahrungen bereits existierender Haushaltsrechner und verdichtete die
Recherchen zu einem \emph{Evaluationsbericht}, aus dem eine
\emph{Anforderungsanalyse} und ein \emph{Partizipationskonzept} abgeleitet
wurden.  In letzerem ist bereits festgehalten, dass Beteiligungskonzepte
übergreifenden Charakter haben müssen und eine allein auf den
Haushaltsplanrechner fokussierte Partizipationslösung zu einer unzulässigen
Engführung der Fragestellungen führt. Diese Position bestätigte sich bei den
Anforderungsaufnahmen in späteren Workshops und Bürgerbeteiligungsphasen im
Rahmen des Projekts.  Zugleich wurde in dieser ersten Phase ein genaueres
Datenmodell für Haushaltsdaten auf der Basis des RDF Data Cube entwickelt und
die vorliegenden Haushaltsdaten für 2014 in dieses Format transformiert.

In einer \emph{zweiten Etappe} (Januar bis Mai 2015) wurde im Rahmen einer
Projektgruppenarbeit im Softwaretechnik-Praktikum auf der Basis der
vorliegenden Anforderungsanalyse der neue Interaktiven Haushaltsplanrechner als
Online-Tool zur Bürgerbeteiligung prototypisch entwickelt.  Mit \emph{Drupal}
kommt dabei ein weit verbreitetes Open Source CMS zum Einsatz, das einerseits
um entsprechende semantische Konzepte für Management und Darstellung von
RDF-basierten Daten und andererseits um angemessene Visualisierungskonzepte zu
erweitern war. Die technischen Voraussetzungen für die Umsetzung des
Partizipationskonzept wurden durch Einbindung und Anpassung eines Drupal-Forums
geschaffen.  Als RDF Data Store kommt eine Instanz von Virtuoso zum Einsatz,
deren SPARQL Endpunkt öffentlich zugänglich ist und damit das Potenzial bietet,
die Plattform in einen größeren Open Data Verbund zu integrieren.

In einer \emph{abschließenden dritten Etappe} (Juni bis September 2015)
arbeitete ein Teil des zweiten Projektteams an der weiteren Konsolidierung des
Prototyps, wobei weitere Anforderungen aus Bürgerbefragungen und
Good-Practice-Analysen aufgearbeitet, priorisiert und im Rahmen eines Redesigns
vor allem des Informationsteils des Interaktiven Haushaltsplanrechners
umgesetzt und integriert wurden.  Außerdem wurde die Datenbasis auf den
aktuellen Stadt-Haushalt 2015/16 aktualisiert. 

In der finalen Version des Haushaltsplanrechners werden die Haushaltsdaten bis
zur vierten Ebene der Schlüsselprodukte aggregiert dargestellt, zu denen auch
Produktsteckbriefe vorliegen. Entsprechende Informationen wurden im Rahmen der
Datenaufbereitung aus den vorliegenden detaillierteren Primärdaten für die
Planwerte der Jahre 2014 bis 2019 aggregiert.

\section{Grundsätzliche Struktur- und Entwurfsprinzipien}

\subsection{Barrierefreiheit}

Beim Design der Website wurde darauf geachtet, diese gemäß BITV 2.0 \cite{bitv}
barrierearm zu gestalten.  Die Entwickler von Drupal und des eingesetzten
CSS-Frameworks Bootstrap legen auf diese Fragen bereits großen Wert, so dass
allein die Wahl der Softwarebasis des Projekts hohe Standards in Bezug auf
Barrierefreiheit sichert. Bei den Anpassungen, die für den Prototyp
erforderlich waren, wurden diese Standards ebenfalls berücksichtigt. 

\subsection{MVC-Architekturmodell}

Für die Kernfunktionen des Projekts, besonders für die Anbindung der
RDF-basierten Datenschicht, kam das MVC-Architekturmodell \cite{MVC} zum
Einsatz.  Eine typische HTTP-Anfrage wird dabei zunächst aufbereitet und an die
Vermittlungsschicht („Controler“) weitergereicht, welche den Auftrag für die
Verarbeitung in der Modellschicht vorbereitet. Die dort zusammengestellte
inhaltliche Antwort wird an den „View“ weitergereicht, um die Daten
entsprechend zu visualisieren (z.\,B.\ in einem Kreisdiagramm).

Zur Verarbeitung der Anfrage in der Modellschicht wird auf die darunter
liegende Datenschicht zugegriffen, die in unserem Fall aus einem RDF Data
Store für die Haushaltsdaten sowie einer MySQL Datenbank für die
Drupal-Funktionalitäten gebildet wird. 

Da in Drupal und unserer Plattform insgesamt intensiv Javascript-Funktionen
einsetzt werden, die auf dem DOM-Baum der Ein- und Ausgabe operieren, wird das
MVC-Muster jedoch oft nicht strikt eingehalten. Eine zentrale Aufgabe des
„Views“ ist zum Beispiel auch die Verwaltung von Formularen und Dokumenten zur
allgemeinen und übersichtlichen Informationspräsentation für die Benutzer.

\subsection{PAC-Architekturmodell}

Drupal selbst orientiert sich stärker am PAC-Architekturmodell \cite{PAC}
(„Presentation-Abstrac"|tion-Control“) als einem spezifischen MVC-Modell, in dem
die Informationsverteilung über die Controler auf spezifische Weise organisiert
ist -- Controler sind verschiedenen Datenabstraktionskonzepten zugeordnet, die
als „Agenten“ bezeichnet werden.  Dieses Architekturmodell wurde bei den
Erweiterungen des Forums-Moduls berücksichtigt, um die Verwaltung des Forums
und der Vorschläge so weit wie möglich über Drupal-interne Funktionen
abzuwickeln.  

\subsection{Semantische Technologien}

Der Haushaltsrechner verwendet auf der Datenebene die semantische Technologie
RDF \cite{RDF} in der Form von RDF Data Cubes \cite{RDFCube}. Das bedeutet,
dass die in unserer Datenbank dargestellten Ressourcen eindeutig bezeichnete
Dinge der Welt sind, die durch Darstellung in Tripel in eine Relation mit
anderen Ressourcen gebracht werden.

Um diese Daten an die Modellschicht zu binden und mit der RDF Datenbasis über
deren SPARQL Endpunkt zu kommunizieren, wurde die \emph{BorderCloud SPARQL 1.1
  PHP Bibliothek}~\cite{BorderCloud} eingesetzt.  Sämtlicher diesbezüglicher
Code einschließlich der zur Verbindung mit dem SPARQL-Endpunkt benötigten
BorderCloud Bibliothek ist im \emph{Theme}-Ordner \texttt{data} abgelegt.

\section{Frontendgestaltung -- Informationsteil}

\subsection{Allgemeiner Aufbau}

\subsubsection*{Abstraktionsebenen}

Die Präsentation der Daten erfolgt über verschiedene Abstraktionsebenen, die
als verschiedene Ringe in der Diagrammvisualisierung präsentiert werden. Die
Abstraktionsebenen haben den Zweck, die Menge an Daten übersichtlich zu
präsentieren und ein intuitives Verständnis sowohl der Größenverhältnisse als
auch der Verteilung der Ausgaben und Einnahmen zu gewinnen. Die
Abstraktionsebenen orientieren sich hierbei an den Produktebenen des
Haushaltsplans der Stadt Leipzig. Hierbei sollten möglichst alle
Hierarchieebenen bis zum kleinsten Element verfügbar sein.

\subsubsection*{Designkonzept}
Das Aussehen der Website des interaktiven Haushaltsrechners der Stadt Leipzig
orientiert sich stark am grundlegenden Design des Bundeshaushaltsrechners und
wurde an das Look and Feel der Webseiten der Stadt Leipzig angepasst.  Zum
Einsatz kam eine leichte Anpassung des CSS-Frameworks \textit{Bootstrap}
\cite{Bootstrap} und dessen Portierung für Drupal~7 sowie die
Javascript-Bibliothek D3 für die Diagrammdarstellungen.

Wie beim Bundeshaushaltsrechner werden die Daten in Form von Kreisdiagrammen
angezeigt.  Zusätzlich werden Produktbereiche, die kleine Beträge aufweisen
(weniger als 2\,\% der Gesamtsumme der jeweiligen Ebene), in den Diagrammen zu
einem Segment \textit{Sonstiges} zusammengefasst, das nicht weiter expandiert
werden kann.  In der tabellarischen Darstellung der Daten der jeweiligen Ebene
werden diese kleinen Beträge jedoch einzeln ausgewiesen.

\subsection{Controller-Struktur}

Die zusätzlichen Controller unserer Erweiterung dienen vor allem dazu,
SPARQL-Anfragen an das Model zu übergeben, damit das Model die Daten aus dem
Datacube auslesen kann. Dabei werden entweder die Ausgaben oder die Einnahmen
der Stadt ausgelesen, je nachdem wie der Modus gesetzt ist. Zudem parst der
Controller noch die richtigen Namen der Kostenbeschreibung hinzu.

\subsubsection{EndpointHandler (= Model)}
Im EndpointHandler werden die Klarnamen der Produktnummer hinzugefügt, zudem
wird die URI so gegekürzt, dass sie nur die Beschreibung anzeigt. Dies
geschieht in der Funktion \texttt{sparqling}. Dabei werden die Daten in
JSON-Format geparst. Dieses Format wird für das Diagramm benötigt. Im
Konstruktor der Klasse wird der Endpunkt zu dem Datacube gelegt, welche über
die Variable endpoint festgelegt wird.

\subsubsection{QueryWriter (= Controller)}
In der Variable \texttt{prefix} werden alle benötigten Präfixe für die Abfrage
gespeichert. Der Präfix \texttt{xsd} wird für die Typkonvertierung benötigt,
während \texttt{qb} für die Zeitangaben im Cube benötigt wird.

In der Variablen \texttt{ihr\_uri} wird die URI des Datacube gespeichert. Wird
der Name des Datacube geändert, so muss auch diese Variablen geändert
werden. Dabei bleibt \texttt{PREFIX ihr:} immer gleich, da ansonsten Fehler in
der Abfrage entstehen. 

Das Array \texttt{attribut\_array} zeigt alle Kategorien des Datacube in
chronologischer Reihenfolge an. Diese werden für die Erstellung der Abfrage
benutzt. 

Das Array \texttt{amount} wird für die Unterscheidung von Ausgaben und
Eingaben benutzt. Die Variable \texttt{choose\_amount} setzt den Modus, ob
Einnahmen oder Ausgaben genutzt werden. Diese Variable wird über die Funktion
\texttt{setAmount} entweder auf 0 (Einnahmen) oder 1 (Ausgaben) gesetzt.  

Die Funktion \texttt{firstRing(\$year = null} erstellt den ersten Ring, den
der Nutzer sieht, dabei ist das Jahr optional. 

Die Funktion \texttt{sparqlTransformation(array category, year)} erstellt eine
SPARQL-An"|frage dynamisch auf Grundlage des Pfades, welchen der Nutzer
ausgewählt hat.

\texttt{searchLabel(\$term)} sucht in der Datenbasis nach der Nutzereingabe.
Dabei werden nach Übereinstimmungen gesucht, somit findet er bei der Eingabe
„Wesen“ auch „Gesundheitswesen“. Es werden dabei alle Nummer und das Label
zurückgegeben. Dabei wird die Funktion durch \texttt{searchNumber(\$term)}
unterstützt, welche noch alle Eingaben und Ausgaben gesucht wird.

\texttt{getPath(\$number)} gibt den Pfad zur einer bestimmten Produktnummer
zurück, welche für die Weiterleitung auf das Diagramm.  Die Funktion
\texttt{setFrom(\$year)} ändert die From-Klausel für die Anfrage ab.
\texttt{getFrom()} gibt die From-Klauseln zurück.

\subsection{Suchfunktion}
Die Suchfunktion ermöglicht die Suche von Labels etc.\ in dem RDF-Graphen.  Es
nutzt JS UI für das Autocomplete und printed in php größtenteils JS-Code,
welcher von php-RDF-Anfragen die benötigten Informationen erhält und daraus
eine Tabelle erstellt.  Unter
\texttt{Drupal/themes/[Themename]/templates/page.tpl.php} wird eine Form
eingefügt, welche zur Suche verlinkt ist, damit sie Drupal oben als Suchleiste
anzeigt.

Die Suche selbst ist als normale Seite einzufügen. Als Textformat muss „PHP
Code“ gewählt werden.

\subsection{Diagramme}

Die in der Plattform enthaltenen Kreisdiagramme sind mit der
Javascript-Bibliothek D3 in der Version 3.5.6 \cite{D3} erstellt.  Kenntnisse
im Umgang mit D3 sind erforderlich, um die Arbeitsweise der Datendarstellung
zu verstehen.

\subsubsection{Grundlegendes}

Der Quelltext jedes Diagramms besteht aus drei Teilen: dem HTML-Teil am
Anfang, der die Verankerung in der Seite festlegt, dem JavaScript-Teil für das
Einnahmendiagramm und dem JavaScript-Teil für das Ausgabendiagramm. Dabei
handelt es sich bei dem Ausgabendiagramm um eine fast exakte Kopie des
Einnahmendiagramms, lediglich mit verschiedener Variablen- und
Funktionenbezeichnern.  Deshalb wird im folgenden nur auf das
Einnahmendiagramm im Detail eingegangen. Im HTML-Teil werden die Tabs zum
Umschalten zwischen den verschiedenen Ansichten (Einnahmen, Ausgaben etc.)
angelegt. Weiterhin werden hier verschiedene Behälter für die Diagramme und
dem Infopanel neben den Tabs festgelegt.

\subsubsection{Datenstruktur}
Die Daten für die Diagramme und Tabellen sind in Levels/Layers organisiert.
Entsprechend sind Arrays innerhalb des Quelltextes angelegt. So gehören zum
Beispiel die Daten des ersten Diagrammrings zum Layer1, die Daten des zweiten
zu Layer2 usw.

Das Datenobjekt, in dem die vom Diagramm darzustellenden Informationen
gespeichert sind, heißt \emph{dataset}.  Wichtige Arrays und Datenobjekte
sind:
\begin{itemize}
\item \texttt{dataset} speichert die Daten des Diagramms in der Form
  \begin{quote}\tt
    \{ layer1: [[ProdNr11, Betrag11, ausgeschriebener Name11] {\ldots}\\{}
      [P1n,B1n,aN1n]]\\ layer2: [[P21,B21,aN21] {\ldots} [P2m, B2m, aN2m]]
       {\ldots} \}
  \end{quote}
\item \texttt{selectedArr} 
  \begin{itemize}
  \item speichert die Produktnummern ab, die durch einen Klick vom User
    ausgewählt wurden, um deren Unterkategorien einzusehen
  \item hält sich an die Levelstruktur, ist aber mit 0 beginnend
  \item Form: \texttt{[ProdNr1, ProdNr2, ProdNr3 {\ldots}]}
  \end{itemize}
\item \texttt{pathArr} 
  \begin{itemize}
  \item enthält sämtliche im Diagramm angezeigte Diagramm-Stücke als mit D3
    ausgewählte Path-Objekte 
  \item hält sich an die Levelstrucktur, ist aber mit 0 beginnend
  \item Form: \texttt{[[Path-Objekt11, PO12 {\ldots}, PO1n], [PO21,PO22,
        {\ldots},PO2m] {\ldots} ]}
  \end{itemize}
\item \texttt{selectedIndexArr}
  \begin{itemize}
  \item ähnlich zu texttt{selectedArr}, enthält statt Produktnummern aber
    Path-Objekte
  \item Form: \texttt{[path-Objekt1, PO2, PO3 {\ldots}]}
  \end{itemize}
\end{itemize}

\subsubsection{Wichtige Funktionen}
\begin{itemize}
\item \texttt{clickFunction(d,i,j,k)} 

  Hierbei handelt es sich um die Hauptfunktion der Diagramme. Sie baut neue
  Diagrammringe auf und auch überflüssige Datenstrukturen wieder ab und ist
  oft jeweils an einen bestimmten Ring gebunden, d.\,h.\ ein Diagrammring baut
  den nächsten auf und alle folgenden ab.
  \begin{itemize}
  \item $d$ (optional) Datensatz des vorhergehenden Rings.
  \item $i$ (pflicht) Index des aufgerufen Datenwertes.
  \item $j$ (pflicht) Index des aufgerufen Datensatzes.
  \item $k$ (abhängig) Wenn dieser Parameter gesetzt ist, wird davon
    ausgegangen, dass diese Methode nicht von einem Diagrammring ausgeführt
    wird. Sie muss deshalb eine Produktnummer enthalten.
  \end{itemize}
\item \texttt{cutOffArray(i)} 

  Diese Funktion baut überflüssige Datenstrukturen ab und wird in der Regel
  von der \texttt{clickFunction} aufgerufen.
  \begin{itemize}
  \item $i$ gibt an, bis zu welcher Stufe Datenstrukturen abgebaut werden.
  \end{itemize}
\item \texttt{urlPathDia} 

  Diese Funktion wird stets beim Aufruf der Seite aufgerufen.  Sie prüft, ob
  in der URL Parameter mitgegeben wurden, und baut diesen Parametern
  entsprechend die Diagramme auf.
\end{itemize}
Alle weiteren Funktionen werden nur unterstützend hinzugezogen und dienen
vorrangig der Realisierung von Darstellungsdetails.

\subsubsection{Generelles Vorgehen}
Beim Aufruf der Seite wird zunächst eine Anfrage mit Hilfe von Ajax gestartet,
deren Antwort die Daten für das erste Level enthält. Diese werden mit Hilfe
von verschieden Funktionen in die entsprechenden Arrays und Variablen
eingespeichert. 

Dann wird mit Hilfe von D3 der erste Diagrammring erstellt und mit
Event-Listener-Funktio"|nen versehen (unter anderem die
\texttt{clickFunction}).  Ab da werden weitere Ringe nur über die
\texttt{clickFunktion} aufgebaut, welche weitere Daten über Ajax-Anfragen
erhält. Für die Erstellung der Ajax-Anfragen werden die in
\texttt{selectedArr} gespeicherten Daten ausgelesen und mit der Auswahl des
Users (über die Parameter $i$ und $j$ ermittelbar) kombiniert, um
festzustellen, welche Daten aus der Datenbasis ausgelesen werden müssen. Die
Antwort auf die Ajax-Anfrage muss eine bestimmte Form haben, um lesbar zu
sein:
\begin{quote}  \tt
  '[[KopfzeileName1, KopfzeileName2, KopfzeileName2]\\{}
    [{\dq}Abteilungsnummer1{\dq}, Betrag1, {\dq}Abteilung1{\dq}]\\{}
    [{\dq}An2{\dq},Betrag2,{\dq}B2{\dq}]{\ldots}]'
\end{quote}
Es handelt sich dabei um einen einzigen String, in dem nur eine Arraystruktur
hineingeschrieben wurde.

\subsubsection{Modifikationsvariablen}
\begin{itemize}
\item \texttt{SIZE} legt die Größe des Diagramms fest.
\item \texttt{limit} legt die Anzahl der Diagrammringe fest.
\item \texttt{DiaURL} legt fest, von welcher externen Datei das Diagramm seine
  Daten bezieht.
\item \texttt{steckURL} gibt an, von welcher Datei das Diagramm
  Produktsteckbriefe beziehen kann.
\item \texttt{colorRing}: In diesem Array sind die vom Diagramm verwendeten Farben als
  Hexadezimalzahlen gespeichert.
\item \texttt{grayFac} bestimmt, wie stark die Farbe eines nicht ausgewählten Segments
  einem Grauton angenähert wird.
\item \texttt{whiteFac} bestimmt, wie stark die Farbe eines nicht ausgewählten
  Segments der Farbe weiß angenähert wird.
\end{itemize}    

\subsubsection{Diagramm}
Neben den Tabellen Id's, die im oben genannten HTML-Teil vergeben werden,
benötigt man nur die beiden Funktionen \texttt{drawTable(array,theTable)} und
\texttt{clearTable(theTable)} (leicht umbenannt für das Ausgabendiagramm) zum
Aufbau der Tabelle. \texttt{drawtable} übernimmt einen Array mit den zu
zeichnenden Daten und erstellt daraus in dem angegebenen html-table Container
die Tabelle. \texttt{clearTable} leert den angegebenen Container und sollte
vor \texttt{drawTable} genutzt werden.

\section{Frontendgestaltung -- Partizipationspaket}

\subsection{Forum}
Die erste zentrale Komponente der partizipativen Ebene des „Interaktiven
Haushaltsrechners“ ist das Forum, welches sowohl für allgemeine Diskussionen,
als auch zur Ausarbeitung oder Diskussion von Vorschlägen verwendet werden
kann. 

Das Forum wird durch die Drupal-eigenen Module \textit{Forum} und
\textit{Advanced Forum} implementiert. Die Nutzerverwaltung wird so von Drupal
selbst übernommen und muss nicht extern verwaltet werden. Weiterhin ist es so
einfacher für Informationsebene und Partizipationsebene ein einheitliches
Design zu erstellen, da beiden dasselbe Drupal Theme zu Grunde gelegt werden
kann.

Es ist wie folgt zu strukturieren:
\begin{itemize}
\item Das Forum muss moderiert sein.

  Hierzu werden verschiedene Userrollen im Forum benötigt. Diese werden direkt
  von Drupal verwaltet und mit entsprechenenden Rechten versehen.

\item Es gibt bei der Registrierung einen Klarnamenzwang. Dieser Klarname wird
  nicht öffentlich angezeigt.
\item Es existiert ein Melde-Button zum Melden von Beiträgen. 

  Der Melde-Button wird durch die Drupal-Module \textit{Flag} und \textit{Flag
    Abuse} verwirklicht.

\item Diskussionen zu einem konkreten Vorschlag finden sowohl unter dem
  Vorschlag, als auch im Forum statt. Ein Kommentar unter dem Vorschlag wird
  jedoch im Forum angezeigt und umgekehrt.
\item Die Vorschläge und Kommentare sollen bewertet werden können. Dazu kommt
  das Drupal-Modul \textit{Rate} zum Einsatz.
\item Der Aufbau vom Forum sollte sich an folgenden Bereichen orientieren:
  \begin{itemize}
  \item[1.] Allgemeine Diskussionen
  \item[2.] Vorschläge (Unterteilung in die Kategorien, verknüpft mit den
    Vorschlägen, Diskussion / Fragen im Forum, liken etc beim Vorschlag)
  \item[3.] Konkrete Themen zum Diskutieren (z.B.\ „Wie gehen wir mit
    Investitionen in Leipzig in Zukunft um?“)
  \end{itemize}
\end{itemize}

\subsection{Vorschläge}
Die zweite zentrale Komponente des Partizipationspakets sollte sowohl bei der
jeweiligen Kategorie im Informationsbereich zur Verfügung stehen, als auch in
einem extra Formular. Nutzer sollen so in der Lage sein, Vorschläge zur
Verbesserung des städtischen Haushalts einreichen zu können. Dazu soll es in
der Informationsebene zu jeder Hierarchieebene des Haushalts einen Button geben
der direkt auf ein Formular verweist über das ein Thread im Forum erstellt
wird. Jeder Vorschlag wird in eine Kategorie eingeteilt, die der höchsten
Abstraktionsebene entspricht. Die Kategorien, welche keine Möglichkeit zur
Mitbestimmung liefern, sollten auch nicht auswählbar sein. Es sollte die
Möglichkeit bestehen, Vorschläge zu bewerten und zu kommentieren. Die
Vorschläge und die Diskussion dazu sind mit dem Forum verbunden.

\subsection{Bürgereinwände}

Mit Blick auf die bis zuletzt unklare Anforderungslage zum Thema
„Bürgereinwände“ sowie die generellen Unwägbarkeiten, ob überhaupt ein Forum
wie im Prototyp vorgeschlagen zum Einsatz kommt, wurde nur die folgende
Variante eines \emph{einfachen Bürgereinwands} umgesetzt:
\begin{itemize}
\item Angemeldete Benutzer können einzelne Vorschläge über einen Knopf als
  (einfachen) Bürgereinwand einreichen. 
  
  Ein spezifischer Registrierungsprozess, wie für \emph{qualifizierte
    Bürgereinwände} gefordert, findet dabei nicht statt. Moderatoren haben
  Zugriff auf die für sie durch die Administration freigeschalteten
  persönlichen Daten der Einreicher, welche diese zu ihrem Account übermittelt
  haben, und können auf diesem Weg Benutzerinformationen einsehen und
  ggf. weitere Informationen über die Kontaktfunktionen der Plattform
  anfordern. 

\item So markierte Vorschläge werden zusammen mit dem Verweis auf den
  Einreicher in einer drupalinternen Liste aggregiert, die nur in der
  Moderatorensicht zugänglich ist.

  Damit kann erfasst werden, welche Vorschläge wie oft als Bürgereinwände
  eingereicht wurden.

\item Reaktionen der BEBS auf einzelne Vorschläge (Verwaltungsstandpunkt)
  können als Beitrag im entsprechenden Vorschlagsthread veröffentlicht werden.
\end{itemize}
Ausführungen zu möglichen Erweiterungen dieses Konzepts sind im Dokument
„Anforderungserhebungen“ zu finden.

\section{Datenmodell}

Die vom Auftraggeber zur Ver"|fügung gestellten Daten für unser Projekt
bestehen aus mehreren Excel-Dateien, die per dynamischer Listenausgabe aus SAP
exportierte relevante Daten zum Doppelhaushalt 2015/16 enthalten.  In Zuge des
Redesigns wurden diese Daten in RDF Cube Serien transformiert.  Weiterhin
wurden aus den Daten der Produktbaum, das Produktmodell sowie die
Produktbeschreibungen der Schlüsselprodukte extrahiert.  Ergänzend enthalten
die Daten einen RDF Graphen \emph{Config}, der die Jahreszuordnung der
eingesetzte RDF Data Cube Serie näher beschreibt, sowie eine Beschreibung des
Dataset-Formats der RDF Data Cube Serie. 

Details hierzu sind im Dokument \emph{Designprinzipien des IHR-15 RDF Data
  Stores} zu finden. 

\section{Dokumentationskonzept}

\subsection{Interne Dokumentation}
Als internes Dokumentationsformat wird einerseits für in PHP programmierte
Elemente das Software-Dokumentationswerkzeug \emph{PHPDoc} verwendet. Diese
Dokumentation kann mit entsprechenden Tools wie z.\,B.\ \emph{phpDocumentor}
aus dem Quelltext generiert werden. Andererseits wird für die Dokumentation
des Diagramm-Quellcodes \emph{JSDoc} verwendet.

\section{Glossar}

\paragraph{Drupal.} 
Drupal~7 ist ein freies (GPL) und kostenloses Content-Management-Framework auf
der Basis von PHP. Es besteht aus einem Core, der aus anderen
Content-Management-Systemen bekannte Funktionen liefert, und Modulen, mit
denen man zusätzliche Funktionen implementieren kann.

\paragraph{Forum.} 
Ein Internetforum dient als virtueller Platz zum Austausch von Gedanken,
Meinungen und Erfahrungen. Ein großer Vorteil dieser Kommunikationsform ist
deren Asynchronität, d.\,h.\ dass Diskussionsteilnehmern ermöglicht wird,
zeitversetzt zu antworten. Üblicherweise werden zur Unterhaltung sogenannte
\textit{Threads} erstellt (= Themen), in denen einzelne Diskussionsstränge zu
einem übergeordnetem Thema gebündelt sind.

\paragraph{Model-View-Controller.} 
Als eines der beliebtesten Architekturmodelle der objektorientierten
Programmierung ermöglicht das MVC-Muster \cite{MVC} eine klare Rollenteilung
im Programm in die drei Einheiten Datenmodell (\textit{model}), Präsentation
(\textit{view}) und Programmsteuerung (\textit{controller}). Der „Observer“
(controller) nimmt hierbei eine zentrale Rolle ein -- er ermöglicht es, auf
Veränderungen in den zu beobachten Objekten zu reagieren. Diese werden durch
das „Model“ und den „View“ beschrieben.

Das „Model“ bietet Zugang zu zentralen Daten des Programms und Funktionen, um
diese abzurufen oder zu verändern. Diese Daten werden dann in dieser
Architektur vom „View“ dem Benutzer zugänglich gemacht. Auf Benutzereingaben
reagiert von hier an wieder der „Observer“ und kann so die Daten abhängig vom
Programmverlauf ändern.

\paragraph{Presentation-Abstraction-Control.} 
ist ein von Drupal verwendetes Architekturmodell \cite{PAC}, das MVC ähnelt.
Es teilt ein System in die Komponenten „Presentation“, „Abstraction“ und
„Control“ ein.  „Presentation“ ist dabei für die Darstellung der Daten
zuständig, „Abstraction“ für die Verwaltung der Daten und „Control“ für
Flusskontrolle und Kommunikation zwischen den beiden anderen Komponenten.

\paragraph{RDF.}	
Das Resource Description Framework \cite{RDF} als Konzept erweitert die
Möglichkeit des Webs, Inhalte miteinander zu verbinden. Hierbei wird versucht,
nach semantischen Zusammenhängen zu ordnen. Die Aussagen die die eigentlichen
Daten in RDF sind, betreffen sogenannte Ressourcen. Diese Ressourcen sind
eindeutig bezeichnete Dinge der Welt, die durch Tripel oder Graphen in eine
Relation mit anderen Ressourcen gebracht werden. Nicht nur durch das mit RDF
im Zusammenhang benutzten Vokabular ähnelt dieses Modell stark an die
klassischen Modelle wie UML-Diagramme oder Entitiy-Relationship-Modelle, auch
die Modellierungsweisen sind stark miteinander verknüpft, wodurch RDF als
Grundansatz des Semantischen Webs an Bedeutung gewinnt.

\paragraph{RDF Data Cube.} 
RDF Data Cube \cite{RDFCube} ist ein spezielles Vokabular zur Beschreibung von
Daten im RDF-Standard. Durch das Bereitstellen von einer beliebigen Anzahl von
Dimensionen ist das Format vor allem zur Beschreibung von statistischen Daten
oder Messdaten geeignet, die nach einem festen Schema mehrfach erhoben werden.

\paragraph{SPARQL.} 
SPARQL als rekursives Akronym für \textit{SPARQL Protocol And RDF Query
  Language}) \cite{SPARQL} ist eine Abfragesprache für RDF, d.\,h.\ eine
semantische Abfragesprache für Datenbanken, mit der man im RDF-Format
gespeicherte Daten aufrufen, durchsuchen und manipulieren kann.

\paragraph{Triplestore.}	
Ein Triplestore ist eine spezielle Datenbank zur Speicherung und Abfrage von
Tripeln durch semantische Abfragen. Ein Daten-Tripel besteht aus Subjekt,
Prädikat und Objekt, z.\,B.\ „Franz kennt Fritz“.

\paragraph{BorderClound SPARQL 1.1 PHP Bibliothek.} 
BorderCloud \cite{BorderCloud} ist eine Programmierbibliothek für PHP, die
Funktionen zum Senden von Anfragen und Verwalten von Antworten an einen
Tripelstore-Endpunkt zur Verfügung stellt.

\paragraph{Bootstrap-Framework.} 
Bootstrap \cite{Bootstrap} ist ein Open-Source Framework für HTML, CSS und
JavaScript zur Erstellung von Webseiten.


\begin{thebibliography}{xxx}\raggedright
\bibitem{bitv} BITV 2.0.  \url{https://github.com/BorderCloud/SPARQL}.
\bibitem{Bootstrap} Bootstrap-Framework.  \url{http://getbootstrap.com/},
  \url{https://www.drupal.org/project/bootstrap}. 
\bibitem{BorderCloud} BorderCloud SPARQL 1.1 PHP Bibliothek.
  \url{https://github.com/BorderCloud/SPARQL}.
\bibitem{D3} Javascript-Bibliothek D3. \url{http://d3js.org/}.
\bibitem{Drupal} Drupal 7. \url{http://de.wikipedia.org/wiki/Drupal}.
\bibitem{MVC} MVC Konzept.
  \url{http://de.wikipedia.org/wiki/Model_View_Controller}.
\bibitem{PAC} PAC Konzept.
  \url{http://de.wikipedia.org/wiki/Presentation-abstraction-control}.
\bibitem{RDF} RDF.
  \url{http://en.wikipedia.org/wiki/Resource_Description_Framework}.
\bibitem{RDFCube} RDF Data Cube.  \url{http://www.w3.org/TR/vocab-data-cube/}.
\bibitem{SPARQL} SPARQL. \url{http://en.wikipedia.org/wiki/SPARQL}.

\end{thebibliography}

\end{document}
