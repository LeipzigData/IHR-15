\documentclass[11pt,a4paper,twoside]{article}
% Encoding lokal, da manche Editoren danach schauen
\usepackage[utf8]{inputenc} 
\usepackage{ihr-15}

\newcommand{\kommentar}[1]{\begin{quote}\textbf{Kommentar:} #1 \end{quote}}

\title{Zusammenfassung des Bearbeitungsstands\\[3pt] der Anmerkungen und
  Auf"|träge\\[3pt] aus den projektbegleitenden Anforderungserhebungen} 

\author{Projekt „Nachhaltige Stadtfinanzen“} 

\date{Version vom 7. Oktober 2015}

\begin{document}
\maketitle
\tableofcontents
\thispagestyle{empty}
\seitezwei
\newpage

\section{Vorbemerkung}

\textbf{Ziel} des Projektbaustein „Interaktiver Haushaltsplanrechner Leipzig
2015“ im Rahmen des Vorhabens „Nachhaltige Stadtfinanzen – Akzeptanzsteigerung
der bürgerschaftlichen Beteiligung an der Haushaltsplanung“ war es,
\begin{quote}
  ein neues Online-Tool zur Bürgerbeteiligung im Rahmen der Haushaltsplanung
  prototypisch zu entwickeln, welches sich an dem in der zweiten Phase des
  Gesamtprojekts zu verabschiedenden Partizipationskonzept und den Bedürfnisse
  der Zielgruppen orientiert.

Die dafür in maschinenlesbarer Form durch das Dezernat für Finanzen der Stadt
Leipzig zur Verfügung gestellten Haushaltsplandaten eines Referenzjahres waren
dazu gemäß eines ebenfalls zu entwickelnden Datenstrukturierungskonzeptes
zielgruppenspezifisch aufzubereiten sowie Anforderungsaufnahme und Entwurf eng
mit den Verantwortlichen des Gesamtprojekts abzustimmen.
\end{quote}
In einem Meilenstein im Mai 2015 (so die ursprüngliche Planung, später
verschoben auf Juli 2015) sollte das prototypische Online-Tool präsentiert,
daran anschließend weiter konsolidiert und zum Projektende an die Stadt Leipzig
übergeben werden. Die Integration dieses prototypischen Online-Tools in die
städtische Infrastruktur und weitere Anpassungen an die Bedürfnisse der
Stadtverwaltung waren nicht Gegenstand des Projektes. Dies wurde explizit als
Aufgabe der städtischen IT-Beauftragten festgehalten.

Methodisch wurden die im Projektbaustein auszuführenden Arbeiten im Rahmen von
Praktikumsaufgaben in die Lehre in zwei Bachelormodule am Institut für
Informatik der Universität Leipzig integriert und in diesem Kontext
\begin{quote}
  im Wintersemester 2014/15 ein Leistungsumfang von etwa 750\,h (entspricht 95
  Personentagen bzw.\ 5{,}3 Personenmonaten\footnote{Dieser Umrechnung ist die
    übliche Umrechnungseinheit von 18 Personentagen pro Personenmonat zu
    Grunde gelegt.} und im Sommersemester im Rahmen des SWT-Praktikums ein
  Leistungsumfang von etwa 900\,h (entspricht 115 Personentagen bzw.\ 6{,}4
  Personenmonaten) sowie in der darauffolgenden Konsolidierungsphase ab Juni
  2015 noch einmal etwa 325\,h (entspricht 40 Personentagen bzw.\ 2{,}3
  Personenmonaten) studentischer Entwicklungsarbeit erbracht.
\end{quote}
Damit stehen allein im Rahmen dieses Projektbausteins 14 Personenmonate
studentischer Entwicklungsleistung zu Buche, wobei die erforderliche
konzeptionelle und inhaltliche Begleitung durch Prof. Gräbe (4\,h pro Woche
über die gesamte Projektlaufzeit) sowie von Philipp Glinka und Matthias
Redlich (jeweils 2\,h pro Woche über die gesamte Projektlaufzeit) noch nicht
mit eingerechnet ist.

\subsection{Rahmenbedingungen} 

Die eingesetzte Methodik und Verfahrensspezifik der Umsetzung erforderte eine
terminlich enge Bindung an die Taktung der universitären Lehrprozesse.  Auf
die mit der verspäteten Bewilligung des Gesamtprojektes „Nachhaltige
Stadtfinanzen“ verbundenen erheblichen Verzögerungen und mehrfachen
Verschiebungen von Rahmenterminen im Ablauf des Gesamtprojekts konnte deshalb
nur beschränkt reagiert werden. 

Da frühzeitig absehbar war, dass wichtige Anforderungserhebungen des
Gesamtprojektes erst im März 2015 vorgelegt werden können, wurden Anpassungen
im anvisierten Ablauf vorgenommen. Gemeinsam mit den Projektverantwortlichen
entwickelte ein interdisziplinär zusammengesetztes studentisches Team deshalb
im Wintersemester 2014/2015 eine erste (vorläufige) Analyse, die als
Ausgangspunkt für die weitere Arbeit genutzt wurde. Die Ergebnisse
(Evaluationsbericht, Anforderungsanalyse und Partizipationskonzept) erwiesen
sich für die weitere Arbeit im Projektbaustein als tragfähig. Wesentliche
Anforderungen, die auf Grund der Umfragen und Workshops aufgestellt wurden,
waren so bereits angedacht und ließen sich dadurch später in die
Anforderungsdokumente integrieren.

Dessen ungeachtet stellte dies aber eine besondere Herausforderungen an die
Projektorganisation: Denn einerseits galt es die Konzeption und die
Entwicklung wesentlicher Teile der Software voranzubringen; andererseits lagen
detaillierte Anforderungen und Vorgaben der Prozessbeschreibungen insbesondere
der partizipativen Elemente noch gar nicht vor. Im Projektverlauf musste
deshalb auf neue Anforderungen, geänderte Akzente und Schwerpunkte durch agile
Ansätze reagiert werden können. In diesem Zusammenhang haben sich der
Rückgriff auf die
SCRUM-Entwicklungsmethodik\footnote{\url{https://de.wikipedia.org/wiki/Scrum}}
und das Führen eines kontinuierlich fortgeschriebenen und mit den Partnern in
der Leipziger Stadtverwaltung abgestimmten Anforderungs- und
Aufgabenprotokolls bewährt.

Mit dieser strukturierten Vorgehensweise und dem Priorisieren und Bündeln von
Anforderungen ist es insbesondere in der letzten Projektphase (Juni bis August
2015) trotz hohem zeitlichen Druck gelungen, die Anforderungen aus den
Analysen in den Rahmen einer \emph{prototypischen} Lösung zur
Weiterentwicklung des Interaktiven Haushaltsplans einzuordnen und alle als
wichtig gekennzeichneten Anforderungen auch umzusetzen.

Die verfügbaren Ressourcen wurden in dieser letzten Projektphase auf die
prototypische Umsetzung aller wichtigen \emph{funktionalen} Anforderungen
(Umstellung des Frontends auf das überarbeitete Datenformat,
Auswahlmöglichkeit verschiedener Jahre, Jahresvergleich, Datenexport als CSV,
semantische Suche über den Beschreibungen der Schlüsselprodukte,
Bürgereinwand) konzentriert. Für eine intensive Test- und Bugfixing-Phase
reichte die Zeit nicht aus. \textbf{Dies ist bei der Planung der Vorbereitung
  des Prototyps auf einen zukünftigen produktiven Einsatz zu
  berücksichtigen. }

\subsection{Weiterführende Arbeiten} 

Seitens der Stadt wurde betont, dass das Gesamtprojekt dazu dient, eine
qualifizierte Entscheidungsgrundlage hinsichtlich eines möglichen Relaunchs
des Interaktiven Haushaltsplans zu schaffen.  Von Seiten der Stadt wurde dies
am 07.05.2015 wie folgt zusammengefasst:
\begin{quote}
  Ausgangspunkt für uns war und ist die Frage, ob der Interaktive
  Haushaltsplan das geeignete Medium ist, um möglichst viele Interessenten
  über den Haushalt der Stadt Leipzig zu informieren, oder eben nicht.
\end{quote}
Es ist übliche Praxis, vor einer mit deutlichen Kosten verbundenen
Entscheidung eine derartige Machbarkeitsstudie in Auftrag zu geben, die nicht
nur die Einsatzbedingungen und -szenarien im Detail abklopft, sondern auch
eine prototypische technische Lösung umfasst. Eine Entscheidung darüber, ob
und ggf.\ unter welchen Rahmenbedingungen ein Relaunch des Interaktiven
Haushaltsplans stattfindet und inwiefern der Prototyp als Grundlage genutzt
wird, fällt erst nach Abschluss des Gesamtprojektes.

Eine Anforderungserhebung über mögliche konkrete Einsatzbedingungen ist vor
einem solchen prinzipiellen Votum allerdings nur in sehr beschränktem Maße
möglich.  Zur Überführung einer im Rahmen einer solchen Machbarkeitsstudie
erstellten prototypischen technischen Lösung in den Regelbetrieb sind durch
den künftigen Betreiber in jedem Fall umfangreichere Einarbei"|tungs- und
Anpassungsarbeiten erforderlich. Unter den gegebenen konkreten Bedingungen
hängt deren Umfang vom jeweiligen Ausroll-Szenario ab. Aus unserer Sicht sind
hier drei prinzipielle Szenarien möglich:
\begin{itemize}
\item[1.] Die strategischen Architekturentscheidungen der Entwickler des
  Prototyps über den Einsatz des OpenSource CMS \emph{Drupal} und des CSS
  Frameworks \emph{Bootstrap} werden beibehalten und der Haushaltsplanrechner
  wird nur lose in die Webarchitektur von \texttt{leipzig.de} eingebunden.

  Schwerpunkt der Anpassung wäre in diesem Fall
  \begin{itemize}
  \item die Transformation der Daten einer durch das Dezernat für Finanzen der
    Stadt Leipzig zur Verfügung zu stellenden \emph{konsolidierten} Datenbasis
    in das Zielformat,
  \item Tests und Bug fixing im Rahmen eines umfangreicheren Probebetriebs
  \item sowie ggf.\ die Umsetzung weiterer Funktionen, die sich durch ein
    übergreifendes Informations- und Beteiligungskonzept ergeben.
  \end{itemize}
  Geschätzter Beratungs- und Unterstützungsaufwand durch das Entwicklerteam:
  2~Personenmonate auf Mitarbeiter-Niveau für Transformation, Tests und
  Bugfixing, bedarfs"|abhän"|gig ggf.\ mehr für die Erweiterung der
  Funktionalität.

  Die Schätzung geht davon aus, dass der Betreiber umfangreiche Erfahrungen mit
  dem Einsatz von Drupal hat und damit über konkrete Architekturfragen
  kompetent verhandeln kann. 
\item [2.] Der Einsatz des OpenSource CMS \emph{Drupal} wird beibehalten, aber
  das CSS Framework Bootstrap durch das stadteigene CSS Framework ersetzt. Dies
  ist durch den (weiteren) Ausbau eines stadteigenen Drupal-Themas durchaus
  möglich, erfordert aber umfangreiche CSS-Anpassungsarbeiten, mit denen eine
  auf Webdesign spezialisierte Firma zu beauftragen wäre.  Dieser Aufwand fiele
  zusätzlich zu 1. an.

  Geschätzter zusätzlicher Beratungsaufwand der Webdesign-Firma durch das
  Entwicklerteam: 1 Personenmonat.
\item [3.] Der Prototyp wird nach Typo3 migriert, um den Haushaltsplanrechner
  enger in die Webarchitektur von \texttt{leipzig.de} einzubinden.

  Hierfür wären umfangreiche Reimplementierungen auszuführen und insbesondere
  der gesamte Forumsbereich neu zu entwickeln, da der Prototyp einen
  Forumsbereich aus dem Drupal-Kern eingesetzt und für die spezifischen
  Projektanforderungen angepasst hat.

  Die Architektur des Informationsteils des Prototyps ist dagegen nur lose in
  Drupal integriert und lässt sich mit mittlerem Aufwand auf ein anderes
  Framework auf der Basis von PHP und Javascript migrieren. 

  Da auf wesentliche konzeptionelle Vorarbeiten der Entwickler des Prototyps
  zurückgegriffen werden kann, sollte eine solche Migration des
  Informationsteils mit einem Aufwand von 2 bis 4 Personenmonaten
  IT-Entwicklerleistung möglich sein, wenn IT-Entwickler zum Einsatz kommen,
  die sich sowohl mit PHP und Javascript als auch mit Typo3 auskennen.  

  Die prinzipielle Realisierbarkeit einer Migration auf Typo3 wäre vorab zu
  prüfen. 
\end{itemize}
Entsprechende Anpassungs-, Beratungs- sowie Weiterentwicklungs- und
Betreuungsleistungen könnten durch die Arbeitsgruppe „Angewandte Semantische
Technologien“ am Institut für Informatik der Universität Leipzig in einem
Folgeprojekt bzw.\ dauerhaft erbracht werden. Eine entsprechende
Aufwandsentschädigung wäre in diesem Fall zu vereinbaren.

\subsection{Dieses Dokument} 

Dieses Dokument enthält hinsichtlich der Erarbeitung und Umsetzung eines
prototypischen Online-Tools eine aktualisierte und kommentierte
Zusammenfassung des Stands der Bearbeitung der Anmerkungen und Aufträge aus
den projektbegleitenden Anforderungserhebungen, wie sie in verschiedenen
Anforderungsdokumenten aus dem Gesamtprojekt an uns herangetragen wurden.
Diese Anforderungen wurden systematisiert, zu Anforderungsblöcken
zusammengefasst, durch das Team auf Konsistenz und Machbarkeit geprüft,
priorisiert und in die Entwicklungsplanung eingeordnet.

Dieses Anforderungsdokument wurde kontinuierlich fortgeschrieben und
regelmäßig mit den Projektpartnern in der Stadtverwaltung (insbesondere dem
Dezernat für Finanzen und der Koordinierungsstelle für Bürgerbeteiligung
„Leipzig weiter denken“) abgeglichen.

In dieser finalen Version ist der Stand der Umsetzung der einzelnen
Anforderungen genauer beschrieben bzw.\ begründet, warum einzelne Anforderungen
unter den gegebenen Rahmenbedingungen für eine prototypische Implementierung
außer Betracht blieben bzw.\ bleiben mussten. 

\textbf{Zusammenfassend kann eingeschätzt werden:}
\begin{itemize}\itemsep0pt
\item Das mit dem Projektbaustein verfolgte \textbf{Ziel}, ein neues
  Online-Tool „Haushaltsplanrechner“ auf der Basis moderner
  informationstechnischer Konzepte aus dem Bereich semantischer Technologien
  und Open Data \textbf{prototypisch} umzusetzen, wurde vollinhaltlich
  erreicht.  Im Projekt wurde eine mit semantischen Technologien angereicherte
  Lösung eines Interaktiven Haushaltsplans entwickelt, die mit ihrer Kopplung
  semantischer Konzepte des Web 2.0 an verbreitete Visualisierungsstandards
  über die Stadtgrenzen hinaus wegweisend ist.

\item Die Flexibilität der software-technischen Architekturkonzepte wird auch
  daran deutlich, dass in der letzten Projekt"|etappe noch eine Reihe von
  funktionalen Anforderungen umgesetzt werden konnten, die in dieser
  Deutlichkeit erstmals im Workshop am 08.07.2015 formuliert wurden und ein
  umfangreicheres Redesign sowohl der Datenschicht als auch des Frontends
  erforderlich machten.
\item Die verfügbare Datenbasis bleibt nach wie vor fragil.  Bis zum
  Projektende konnten fundamentale Unstimmigkeiten in der Interpretation der
  durch das Dezernat für Finanzen der Stadt Leipzig zur Verfü"|gung gestellten
  Haushaltsdaten nicht ausgeräumt werden, so dass die mit dem Prototyp
  ausgelieferten Daten \textbf{nur kursorisch die Funktionalitäten des
    Haushaltsplanrechners demonstrieren, nicht aber belastbare Informationen
    zur Haushaltsplanung der Stadt Leipzig wiedergeben.}
\item Die Umsetzung bleibt auch insoweit \textbf{prototypisch}, als dass zwar
  intensive interne Tests seitens des Projektteams durchgeführt wurden, aber
  mit den Auftraggebern kein sinnvolles Test- und Abnahmeverfahren im Rahmen
  der Projektlaufzeit vereinbart werden konnte, so dass die gesamte
  Softwarebasis nur „as is“ übergeben wird.  Zum nachhaltigen Einsatz des
  Prototyps ist ein Betreiberkonzept erforderlich, das erst nach einer
  grundlegenden politischen Weichenstellung zur weiteren Nutzung des Prototyps
  entwickelt werden kann.
\end{itemize}

\section{Betriebskonzept des Haushaltsplanrechners}

Die folgenden Ausführungen gehen davon aus, dass die im Rahmen der
projektinternen Anforderungserhebungen entwickelten konzeptionellen und
begriff"|lichen Vorstellungen und insbesondere die Ausführungen zu engen und
weiteren Partizipationsszenarien im Partizipa"|tions"|konzept bekannt sind,
siehe dazu die \emph{Anforderungsanalyse} und das \emph{Partizipationskonzept}
in den Projektunterlagen sowie den Projektbericht.

Es wird von folgendem \textbf{Rollenkonzept} ausgegangen: Admins
(Stadtangestellte), Moderatoren (Stadtangestellte und entsprechend autorisierte
aktive Bürger/innen), Forumsnutzer (dürfen im Forum neue Threads anlegen),
Benutzer (registrierte Nutzer/innen), Besucher/innen (nicht registriert,
agieren anonym).

Die Informationsfunktionen des Prototyps können anonym genutzt werden, für die
Bewertungsfunktion kann durch den Betreiber eingestellt werden, ob dies anonym
möglich sein soll oder eine Registrierung erforderlich ist.  

Alle weiteren Funktionen setzen eine Registrierung, Zuweisung einer
entsprechenden Rolle und Authentifizierung durch Login voraus, wofür
Standard-Drupal-Prozesse genutzt werden. Auf Admin-Ebene bestehen hier
umfangreiche Konfigurationsmöglichkeiten durch den Betreiber.

\section{Anonym nutzbare Funktionalitäten des Frontends}

\subsection{Erscheinungsbild und Layout}

Das Erscheinungsbild unseres Prototyps orientiert sich am Interaktiven
Haushaltsplan des Bundes. Im weiteren Verlauf der Entwicklung, insbesondere in
den Workshops, wurden folgende Kritikpunkte an der Darstellung geäußert:
\begin{itemize}\itemsep0pt
\item Einstiegsseite ist optisch nicht ansprechend genug.  Verwendete Fotos
  könnten ansprechender sein und auch unter dem Aspekt Bürgerbeteiligung als
  nur unter dem Aspekt Finanzen/Business ausgewählt werden.  Visualisierung
  durch Bilder mit Menschen. 

\kommentar{Die Startseite wurde an das Look and Feel des Bundesrechners
  angepasst. Die Einstiegsseite wurde durch den Einbau der Grafik eines
  aufgeschlagenen Buchs umgestaltet.  Die verwendeten Bilder und Grafiken sind
  unter der CC Zero Lizenz verfügbar.

  Mit Blick auf die noch ausstehende grundsätzliche Entscheidunge zum weiteren
  Einsatz des Interaktiven Haushaltsplans werden weitere Anpassungen des
  Designs den Webdesignern der Stadt überlassen. }

\item Ringe zu dick, Farben teilweise so, dass man die weiße Schrift der
  Zahlenangaben nicht gut erkennen kann. Farben stringenter mit
  Wiederkennungswert wählen: z.B. Produktbereich Sport in dunkelrot,
  Produktgruppe in rot und einzelnen Produkte in Schattierungen, falls
  möglich.  Gestaltung und Layout: Farben und Größen der Beschriftung ändern
  sich bei Kreisdiagramm.

\kommentar{Im Zuge des Redesigns in der letzten Projektphase wurde auch die
  Visualisierung der Ringe überarbeitet und auf das Javascript-Framework D3
  \url{http://d3js.org/} umgestellt.  In dem Kontext wurden diese Anregungen
  noch einmal diskutiert und weitgehend eingearbeitet.}
\end{itemize}
Generell gibt es hier allerdings einen Zielkonflikt mit der Barrierefreiheit
der Seiten, da sich etwa ausgefeilte Farbkonzepte für visuelle Darstellungen
Blinden nicht erschließen.  Angepasste Lösungen für Zielgruppen mit
verschiedenen Arten von Beeinträchtigungen erfordern zielgruppenspezifische
Navigationsstrukturen.  Gute übergreifende Ergebnisse werden mit schlichten,
informativ ausgerichteten Navigationsstrukturen erreicht, was aber im Konflikt
mit der Forderung nach einer „optisch ansprechenden Visualisierung“ steht, mit
der die Bedürfnisse „normaler“ Nutzer in den Vordergrund gerückt werden.

Nur teilweise umgesetzte Kritikpunkte:
\begin{itemize}\itemsep0pt
\item Im Header muss neben dem Stadtwappen der Stadt Leipzig der Titel der
Fachanwendung eingetragen sein. 
\item Es muss die deutsche Sprache verwendet werden (trifft noch nicht
  vollständig für Forum und Feedback-Box zu).
\item Navigation ist bei Smartphone im Header nicht komplett sichtbar
  (abgeschnitten). Darstellung auf mobilen Geräten ist generell zu prüfen
  (Aussagen auch aus dem Alphatest). 
\end{itemize}

\subsection{Informationsseiten}

Nach dem Workshop am 08.07.2015 und dem Alphatest wurde die Visualisierung im
Prototyp noch einmal einem kompletten Redesign unterzogen. Der Inhaltsbereich
besteht nun aus einem Meta-Panel (rechts), in dem (neben weiteren
Informationen) die Stellung des jeweiligen Knotens des Produktgraphen --
getrennt nach Einnahemn und Ausgaben -- sichtbar ist, und einem Infopanel
(links).  Dort kann auch zwischen den Datensätzen verschiedener Jahre
umgeschaltet werden.

Das Infopanel untergliedert sich in mehrere Tabs (Einnahmen, Ausgaben, Tabelle,
Jahresvergleich). In der Ansicht \emph{Tabelle} wird eine Tabelle zum aktuellen
Jahr angezeigt, wie sich die angezeigte Summe auf Kindknoten des aktuell
ausgewählten Knotens aufschlüsselt oder -- für ein Schlüsselprodukt als Blatt
des Produktbaums ohne weitere Untergliederung -- die Produktbeschreibung
angezeigt. Im \emph{Jahresvergleich} werden die Einnahmen und Ausgaben zum
ausgewählten Knoten im Produktbaum für die verfügbaren Jahre tabellarisch
dargestellt.

Die Navigation erfolgt in den Bereichen „Einnahmen“ und „Ausgaben“ getrennt, da
kleinteilige Posten zu einem Posten „Sonstiges“ auf der Seite der Einnahmen und
der Ausgaben verschieden zusammengefasst sein können, also ein Knoten des
Produktbaums im Bereich der Einnahmen sichtbar sein kann, im Bereich der
Ausgaben aber nicht.  Die Navigation ist sowohl über die Ringe als auch die
Tabellen möglich.

\kommentar{Damit wird die Forderung aus dem Workshop am 08.07.2015
  aufgegriffen, den verschiedenen Informationsbedarfen besser Rechnung zu
  tragen.}

Einzelne Anforderungen:
\begin{itemize}\itemsep0pt
\item Bundesrechner hat auf jeder Seite links oben ein Widget mit
  zusammenfassenden Informationen zur gerade dargestellten Produktgruppe.

  \kommentar{Im Redesign als Meta-Panel umgesetzt. }
\item Dem Benutzer sollte eine einfache Navigation zwischen den Einnahmen und
  den Ausgaben eines Produkts oder einer Produktgruppe möglich sein.

  \kommentar{Im Redesign durch die neue Tab-Struktur des Info-Panels
    umgesetzt.}
\item Zur Vereinfachung der Lesbarkeit wäre es positiv, wenn die Zahlen durch
  Komma und Punkte getrennt sowie rechtsbündig angeordnet wären. Zudem wäre es
  hilfreich, wenn die Anteile in Prozentzahlen auch noch in einer Extraspalte
  vermerkt werden könnten.  So könnten die Produkte auch nach Prozentzahlen
  sortiert angezeigt werden.

  Idee: Vereinfachung der Darstellung durch die Umrechnung des Haushalts
  auf einen Bürger.  Prozentuale Verteilung der Einnahmen und Ausgaben
  umrechnen auf einen Bürger und mit entsprechenden Zahlen versehen. Eine
  Extraseite als Beispiel könnte eventuell so aufgebaut werden, mit
  Darstellung der Einnahmen und Ausgaben in 100\% Säulen-Diagrammen.

  \kommentar{Die Division durch die Anzahl der Einwohner Leipzigs gibt nur im
    Jahresvergleich neue Einsichten.  Andere relative Angaben werden dadurch
    nicht beeinflusst.  Entsprechende Erweiterungen der Daten und Funktionen
    sind möglich. }

\item Wichtig sind verständliche Grafiken, die zeigen, wie sich die einzelnen
  Positionen einordnen. Gut wäre, wenn ggf. einzelne Produktgruppen ausgewählt
  und zusammengefasst werden könnten. Eine wünschenswerte Option wäre die
  grafische Darstellung in verschiedenen Varianten (z.B. Jahresverlauf;
  Einordnung zu Vergleichsstädten etc.).

\kommentar{Die Optik des Prototyps und insbesondere die verwendeten grafischen
  Elemente orientieren sich am Haushaltsplanrechner des Bundes, um hier ein
  gemeinsames „Look and Feel“ zu erreichen.  Vergleiche mit anderen Städten 
  sind im Prototyp nicht vorgesehen, da hierfür keine Daten zur Verfügung
  standen.  

  Solche Vergleiche würden auch das Konzept eines Interaktiven
  Haushaltsplanrechners, in dem es primär um Beteiligung an den
  Haushaltsplanungen der \emph{eigenen} Stadt geht, unnötig überladen und
  damit verwässern.

  Es ist jedoch möglich, auf der Basis entsprechender Daten solche Vergleiche
  nach der im Projekt verfolgten Methodik im Rahmen einer Vergleichsplattform
  darzustellen.  Dies könnte in einem Folgeprojekt angegangen werden.} 
\end{itemize}

\subsection{Erläuterungskomponente}

Zum Aufbau einer solchen Komponente ist es erforderlich, entsprechende
Erläuterungsinhalte aus verschiedenen Quellen zu übernehmen.

\begin{itemize}\itemsep0pt
\item Übernahme der Erläuterungsinhalte aus dem bisherigen HH-Rechner.  
  \kommentar{Umgesetzt.}
\item Übernahme der Beschreibungen der einzelnen Produktnummern, die auch in
  der pdf-Version des Haushaltsplans enthalten sind.  

\kommentar{Solche Beschreibungen liegen nur für die Schlüsselprodukte als
  Blätter des Produktbaums vor.  Diese Produktbeschreibungen sind komplett in
  die Plattform integriert.

  In der überarbeiteten Systematik werden die Haushaltszahlen (des
  Ergebnishaushalts) nur bis zu dieser Ebene der Schlüsselprodukte angezeigt.
  Entsprechend werden in der Datenbasis auch nur diese Aggregate vorgehalten. }

\item Die Produktgruppe, Produktuntergruppen, Produkte folgen einer Logik, die
  erklärt werden sollte. Daher wäre es gut, wenn in den Teil „Anleitung“
  ebenfalls die Zusammensetzung der Produktbezeichnung erklärt werden könnte.
  Mögliches Beispiel siehe
  \url{http://www.bundeshaushalt-info.de/anleitung.html}.

  \kommentar{Dieser Zusammenhang ist im neuen Menüpunkt „Haushaltsplanung“
    erläutert.  }
\end{itemize}

Weitere Anforderungen:
\begin{itemize}\itemsep0pt
\item Eine verständlichere Sprache verwenden und Fachbegriffe gut umschreiben,
  eine bürger- und benutzerfreundliche „Übersetzung“ der Thematik.

  Auf der Einstiegsseite sollen mehr Informationen zum Thema der
  Haushaltsplanung zu finden sein als beim ehemaligen „Haushaltsplanrechner“.
  In erster Linie soll der gesamte städtische Haushalt verstehbar sein,
  d.\,h.\ nicht so kompliziert!!! Ggf. hier eine knackige Aufbereitung der
  Übersichts- und Einstiegsvorträge von Herrn Bonew.

  Übernahme von Inhalten von weiteren Erklärkomponenten aus dem alten
  HH-Rechner: „Häufige Fragen“, „Informationen zur Haushaltsplanung“, „Hinweise
  zur Benutzung des HH-Rechners“, „Fragen und Feedback“.

  \kommentar{Die entsprechenden Erläuterungsinhalte aus dem bisherigen
    HH-Rechner wurden übernommen, so dass in diesem Punkt zumindest der
    bisherige Stand wieder erreicht ist.

    Der Interaktive Haushaltsplan kann einfach um Menüpunkte erweitert werden.
    Der weitere Ausbau der Erläuterungskomponente setzt ein entsprechendes
    Informations- und Beteiligungskonzept voraus. Die inhaltliche Erarbeitung
    der Erläuterungen erfordert zudem in den einzelnen Haushaltspositionen
    vielfältige fachliche Kenntnisse und ressortübergreifende Abstimmungen.
    Dies sind verwaltungsinterne Aufgabe, die über den Rahmen dieses Projekts
    hinausgehen.

    Wesentliche Grundvoraussetzungen für den weiteren Ausbau der
    Erläuterungskomponente sind Zuarbeiten, die erst mit einer Entscheidung
    über einen Relaunch des Interaktiven Haushaltsplans zur Verfügung stehen
    werden und damit dem zukünftigen Betreiber überlassen bleiben müssen.}

\item Exportfunktion für die Daten einer Informationsseite.

\kommentar{Eine solche Exportfunktion wurde erst zu einem späten Zeitpunkt des
  Projektes in die Anforderungen aufgenommen, ist prinzipiell über Anfragen an
  den SPARQL-Endpunkt realisierbar und wurde für ganze Jahres"|sätze (etwa 500
  Datensätze pro Jahr bis zur Ebene der Schlüsselprodukte) in den Formaten CSV
  und JSON prototypisch implementiert. }

\end{itemize}

\subsection{Suchfunktion}

Eine Suchfunktion kann nur die textuelle Information auswerten, die im Backend
verfügbar ist.  Hier existieren mehrere Handlungsansätze, die nach wachsender
Komplexität aufgelistet sind. Im Prototyp sind die ersten beiden Ansätze
umgesetzt. 
\begin{itemize}\itemsep0pt
\item [1.] Nutzung der intern in Drupal verfügbaren Suchfunktion.  Diese führt
  eine Volltextsuche auf allen Seiten aus, also in unserem Fall auf den Threads
  des Forumsbereichs.  Damit kann man Treffer im Vorschlagsbereich finden.

  \kommentar{Umgesetzt, setzt allerdings ein gut befülltes Forum voraus.}
\item [2.] Eine SPARQL-basierte semantische Textsuche kann über die
  Informationen ausgeführt werden, die zu den Knoten des Produktbaums verfügbar
  sind.

\kommentar{Nach dem Redesign sind die Produktsteckbriefe von 205 der 282
  Schlüsselprodukte (es fehlen Steckbriefe von Schlüsselprodukte vor allem aus
  den Bereichen 53, 57 sowie 71--76) sowie die Bezeichnungen der 32
  Produktbereiche, 92 Produktgruppen und 127 Produktuntergruppen in die
  Datenbasis integriert.  Auf dieser Grundlage werden mit einer einfachen
  RegEx-Suche relevante Knoten des Produktgraphen zu einem Suchbegriff
  gefunden.}

\item [3.] Die Suchergebnisse von 1. werden in einen zu erstellenden
  Auswertemodul eingespeist, der Links automatisch heraussucht und das Ergebnis
  vergleichbar zu 2. darstellt.
\item [4.] Aufbau eines Tagging-Systems, das durch die Aufzeichnung der
  Suchbegriffe der Nutzer weiter qualifiziert wird. 
\item [5.] Anbindung eines NLP-Analysewerkzeugs, um aus den Volltexten ein
  Taggingsystem werkzeuggestützt zu extrahieren.
\item [6.] Aufsetzen eines nutzergetriebenen Community-Prozesses zur
  Verbesserung der Verschlagwortung.  Nutzer haben die Möglichkeit, eigene
  Schlagworte mit ins System einzubringen. 
\end{itemize}

Anforderungen seitens der Stadtverwaltung:
\begin{itemize}
\item Die Suchfunktion sollte als eine “intelligente Suchmaske” programmiert
  werden, die es auch ermöglicht ähnliche Begriffe zu finden etc.  Wichtig ist
  hier, eine Suchlogik zugrunde zu legen, die sich am allgemeinen Verständnis
  und nicht an Fachtermini orientiert. Zudem sollten auch Wörter erkannt
  werden, wenn sie evtl. falsch geschrieben werden.

  \kommentar{Die Implementierung einer solchen „Intelligenz“ ist sehr
    auf"|wändig und konnte im Prototyp, dessen Schwerpunkt auf der angemessenen
    Strukturierung und Präsentation der Daten lag, nicht in Angriff genommen
    werden. }

\item Mit der Suchfunktion soll ein Glossar verbunden sein, das entweder im
  Text der Seiten über das „Info I” oder „Mouseover” funktionieren sollte.

  \kommentar{Ein solches Glossar kann als Erweiterung eines Schlagwortgraphen
    eines Taggingsystems entstehen, wenn die Schlagworte mit Erläuterungen
    unterlegt werden.  Eine solche Variante bleibt einem möglichen
    Nachfolgeprojekt vorbehalten.

  Das vom Dezernat für Finanzen der Stadt Leipzig gelieferte Glossar
  (19~Begriffe) ist als Menü"|punkt im Frontend integriert und als
  \texttt{skos:Concept"|Scheme} in der Datenbasis verfügbar. Ein darüber
  hinausreichendes Glossar kann leicht auf dieselbe Weise integriert werden. }

\end{itemize}

\section{Frontendfunktionalität für registrierte Nutzer}

\subsection{Registrierung und Datenschutz} 

\subsection*{Stand der Umsetzung der Registrierfunktion} 

\begin{itemize}\itemsep0pt
\item Die Registrierung erfolgt über Drupal-Standardprozesse. 
\item Ein Admin kann dabei konfigurieren, welche Felder obligatorisch sind und
  welche Felder durch den Benutzer später nicht modifiziert werden können. 
\item Die Selbst-Registrierung kann abgeschaltet bzw. durch eine Captcha-Seite
  auf natürliche Personen beschränkt werden.  Im Zuge des Anmeldeprozesses wird
  ein Verifikationslink per Email verschickt.
\item Zusätzlich kann der Admin die Registrierung so konfigurieren, dass neue
  Benutzer explizit vom Admin freigeschaltet werden müssen. 
\item Eine Plausibilitätsprüfung der Angaben in einzelnen Feldern erfolgt
  nicht (automatisch). 
\item Ein Admin kann einzelne Benutzer sperren oder löschen. 
\end{itemize}

\textbf{Anmerkung:} Im Zuge des Alphatests hat sich gezeigt, dass auch
umfangreiche präventive Maßnahmen nicht gegen eine massenhafte Einrichtung von
Spam-Accounts helfen.  In einer Woche wurden über 25\,000 Accounts angelegt
und teilweise über E-Mail-Verifikation aktiviert. Die Spammer ließen sich auch
durch 2 Anti-Spam-Module nicht aufhalten.  Bei der derzeit vorgesehen
Freischaltung durch den Admin kann zwar eine Registrierung verhindert werden,
jedoch führt bereits der Registrierungsversuch zu einem Datenbankeintrag, der
bereinigt werden muss. Darüber hinausgehender Spam-Schutz erfordert in jedem
Fall eine stärkere manuelle Betreuung.  Im Forum der Drupal-Entwickler wird
darauf hingewiesen, dass es derzeit keinen wirksamen Schutz gegen
Spam-Anmeldungen gibt außer der Einstellung „Administrator only“, in der neue
Nutzer zwingend vom Administrator eingetragen werden müssen. Dazu müsste ein
Drupal-externer Registrierungsprozess aufgesetzt oder ein bereits bestehendes
Registrierungsverfahren erweitert und über ein Sign-In-Verfahren an Drupal
angebunden werden.  Zu letzterem Punkt gibt es umfassende
Erfahrungsberichte\footnote{Siehe etwa
  \url{https://groups.drupal.org/node/182004}.} der
Drupal-Entwickler-Gemeinde.

\subsection*{Aufgabenstellung Datenschutz}

Zu datenschutzrechtlichen Aspekten wurden am 07.01.2015 von der
Stadtverwaltung folgende Grundsätze übermittelt:
\begin{itemize}\itemsep0pt
\item Nur die Daten erheben, die benötigt werden.
\item Information was mit den Daten passiert.
\item Die erhobenen Daten nur zum angegebenen Zweck verwenden.
\item Logins und Anmeldungen nur über double-opt-in Lösungen (Anmeldung und
  Bestätigung per Mail-Link), damit niemand für jemand anderes eine Anmeldung
  vornehmen kann.
\item Einbindung von externen Diensten (z.B. facebook) nur über
  2-Klick-Lösungen.  Alle verwendeten Tools zur Nutzungsauswertung müssen
  datenschutzkonform sein (kein google analytics verwenden).
\end{itemize}

\subsection*{Stand der Realisierung}

Anmeldung und Nutzerverwaltung erfolgen mit Standard-Drupal-Funktionen
\emph{innerhalb} des Prototyps. Externe Login-Dienste (Google, Facebook)
werden nicht verwendet, ebenso keine Single-Sign-On-Konzepte. 

Eine Webanalyse kann durch das Drupal-Modul \emph{Statistics
  Counter}\footnote{\url{https://www.drupal.org/project/statistics_counter}}
umgesetzt werden, entsprechende Informationen verlassen dabei den
Drupal-Kontext nicht.
\pagebreak[3]

Im Einzelnen:
\begin{itemize}\itemsep0pt
\item In Drupal kann konfiguriert werden, welche Felder bei der Registrierung
  verpflichtend auszufüllen sind. 
\item Der Prototyp ist so konfiguriert, dass Klarnamenszwang herrscht, also
  neben einem Nickname auch der richtige Name anzugeben ist. Das kann weiter
  durch einen Admin manuell auf Plausibilität geprüft werden.  Weitere
  verpflichtende Angaben können konfiguriert werden. Diese Informationen sind
  für den Benutzer in dessen persönlichem Profil einseh- und änderbar, für
  andere Nutzer bis zur Rolle „Forumsnutzer“ aber nicht sichtbar.  

  Durch einen Admin können einzelne Felder des Benutzerprofils als durch den
  Benutzer nicht änderbar konfiguriert werden.
\item Die Registrierung wird durch einen Bestätigungslink verifiziert, eine
  zusätzliche manuelle Freischaltung durch einen Administrator kann
  konfiguriert werden. Für diese Funktionalität muss auf dem Server ein
  Email-Dienst verfügbar sein.
\end{itemize}

\textbf{Mögliche Weiterentwicklung:} Einbindung von Social Media
Funktionalitäten über ein entsprechendes Drupal-Modul.

\subsection{Gestaltung des Forumsbereichs} 

\subsection*{Stand der Umsetzung} 

Für den Forumsbereich wurde ein Drupal-Forums-Modul eingebunden, womit
wesentliche Elemente der geforderten Grundfunktionalität bereits verfügbar
waren. 

Ein Thread ist eine einfache Drupal-Seite mit einer Reihe von
Drupal-Kommentaren.  Damit stehen für Threads alle Drupal-Funktionen zur
Verfügung, die generell für Seiten verfügbar sind, und für Beiträge die
allgemeinen Drupal-Funktionen für Kommentare zu einer Seite.

\begin{itemize}\itemsep0pt
\item Ein \emph{Thread} ist eine Menge von Beiträgen. 
\item Einem Thread sind \emph{Metainformationen} zugeordnet.
\item Threads können in \emph{Unterforen} aggregiert werden. Als Unterforen
  sind die beiden Bereiche „Allgemeine Diskussion“ und „Offizielles“ (Themen
  der Stadtverwaltung) sowie die 32 Produktbereiche der Haushaltssystematik
  eingerichtet.
\item Ein Thread fasst alle Beiträge zu einem \emph{Topic} zusammen. 
\item Threads werden zum Einreichen und Kommentieren von Vorschlägen verwendet.
\item Threads haben eine Bewertungsfunktion.
\item Es ist möglich, geschlossene Threads zu bewerten, wenn diese sichtbar
  sind.  
\item Threads können als „veröffentlicht“ sowie „oben in Listen“ markiert
  werden.  
\item Ein Thread kann als „Bürgereinwand“ markiert werden.
\end{itemize}
\pagebreak[3]

Beiträge:
\begin{itemize}\itemsep0pt
\item Ein Thread wird mit einem \emph{Initialbeitrag} eröffnet. Alle weiteren
  Beiträge sind Kommentare von Benutzern zum Thema des Threads.
\item Beiträge in einem Thread sind nicht weiter hierarchisch strukturiert,
  direktes Kommentieren anderer Beiträge erfolgt durch Zitieren von Passagen.
\item Ein Beitrag kann vom Autor jederzeit editiert, aber nicht mehr gelöscht
  werden.
\item Beiträge in einem Thread werden chronologisch sortiert nach „neueste
  zuerst“. Es gibt jedoch verschiedene Optionen, die Threads sowie die Beiträge
  in einem Thread auf der Ebene individueller Nutzereinstellungen zu sortieren.
\item Für längere Beiträge werden automatisch Kurzzusammenfassungen angezeigt,
  die mit „read more“ expandiert werden können.
\end{itemize}

\subsection*{Bewertungsfunktionen für Threads} 

Diese ist als einfache $+/-$ Bewertung für Threads umgesetzt:
\begin{itemize}\itemsep0pt
\item Jeder Benutzer kann jeden Thread höchstens einmal bewerten.
\item Der Benutzer kann die eigene Bewertung zu einem späteren Zeitpunkt
  ändern, wenn der Thread noch offen ist.
\item Der Benutzer bekommt die eigenen Bewertungen angezeigt.
\item Probleme können von Benutzern über einen Melden-Button an die Moderation
  weitergeleitet werden. Alternativ steht das Kontaktformular der Plattform
  dafür zur Verfügung. 
\item Bewertungen können anonym vorgenommen werden, wenn dies so vom Admin
  eingestellt ist.
\end{itemize}
Die Art der Bewertung (anonym oder nur Benutzer) kann vom Admin eingestellt
werden. Im Benutzermodus ist es einfach, Mehrfachbewertungen durch dieselbe
Person zu verwalten. Im anonymen Modus wird versucht, Mehrfachbewertungen über
IP-Adressen zu verfolgen, die einen Tag lang gespeichert werden. 

\textbf{Mögliche Weiterentwicklung:} Ein Analysetool für quantitative
Auswertung von Bewertungen durch die Moderatoren.

\subsection*{Anbindung des Forums an den Informationsteil und
  Metainformationen}  

Zwischen Forum und Informationsteil gibt es nur eine lose Verbindung, da als
einziges Strukturierungselement die Einordnung von Vorschlags-Threads in
Subforen zur Verfügung steht, von denen jedes einem definierten Produktbereich
zugeordnet ist.  

Eine solche lose Kopplung ist auch zielführend, da in den Workshops immer
wieder betont wurde, dass sich die meisten Problemlagen im Haushalt, welche
Bürger bewegen, nicht starr in die vorgegebene Produktsystematik einordnen
lassen.  Die Knöpfe „Vorschlag einreichen“ im Informationsteil (einer für
„Ausgaben“ und einer für „Einnahmen“) führen über die jeweilige
Produktbereichs-Information in das zugeordnete Subforum im Forumsbereich.

Neben der Zuordnung zu einem Produktbereich und der Thread-ID können zu einem
Thread weitere Metainformationen ausgewertet werden.  Drupalseitig kann hier
über Erweiterungen insbesondere eine genauere Webstatistik angebunden werden.
Auch ein Tagging-System für Vorschläge wäre denkbar. 

Mit Blick auf die wenig präzisen Vorgaben zum Partizipationskonzept wurde in
diesem Teil des Prototyps nur ein Grundkonzept umgesetzt und der Ausbau der
Funktionalitäten einem künftigen Projekt überlassen. 

\subsection{Moderation des Forumsbereichs} 

\subsection*{Stand der Umsetzung} 

Folgende Funktionalitäten sind verfügbar:
\begin{itemize}\itemsep0pt 
\item Ein neuer Thread kann nur von einem Benutzer in der Rolle „Forumsnutzer“
  angelegt werden.
\item Jeder Benutzer kann in jedem (offenen) Thread kommentieren. 
\item Um den administrativen Aufwand zu minimieren, ist im Prototyp eine minimal
  moderierte Version umgesetzt, die eine Meldefunktion von Missbrauch und
  potentiellen Verstößen an die Moderation und deren Prüfung vorsieht.  

  Dies kann durch einfache Rekonfiguration des eingesetzten Drupal-Forums
  geändert werden. 
\end{itemize}

\subsection*{Moderation von Threads und Beiträgen} 

Da Threads und Beiträge spezielle Drupal-Seiten und Drupal-Kommentare sind,
kann die Moderation auf die entsprechenden Moderationskonzepte für Seiten und
Kommentare in Drupal zurückgreifen. 

Durch die Nutzung dieser Möglichkeiten können Moderatoren
\begin{itemize}\itemsep0pt
\item unter „Einstellungen für Kommentare“ einzelne Threads schließen oder
  diese unsichtbar schalten,
\item Beiträge oder Threads löschen,
\item die Diskussion zu einem Vorschlag beenden,
\item damit manuell Zeitbgrenzungen für Threads realisieren. 
\end{itemize}

Eine Infrastruktur für automatische Zeitbegrenzungen von Threads ist aktuell
nicht umgesetzt.

\subsection*{Meldefunktion für Beiträge} 

Jeder Thread hat einen „Melden-Knopf“, mit dem der Moderation des
Forumsbereichs Beiträge angezeigt werden können, zu denen der Zugang zu
beschränken ist.  Diese Meldungen werden im Backend aggregiert und müssen von
den Moderatoren zeitnah manuell ausgewertet werden.

\subsection{Bürgereinwände} 

\subsubsection*{Rahmenvorgaben für qualifizierte Bürgereinwände} 

Vorschläge, die wirksam in die Haushaltsdiskussion eingehen sollen, müssen die
Form eines \emph{qualifizierten Bürgereinwands} haben. In der
Anforderungserhebung wurde festgestellt, dass ein in den Befragungen
gefordertes leichtgewichtiges Partizipationskonzept mit geringen Zugangshürden
und die Registrierungserfordernisse für die Formulierung eines qualifizierten
Bürgereinwands zueinander im Widerspruch stehen. 

Die Verarbeitung inkl.\ der Prüfung der personenbezogenen Daten, die für einen
qualifizierten Bürgereinwand zu erheben sind, ist ausschließlich über dazu
autorisierte Stellen in der Stadtverwaltung möglich. Bürgereinwände können
deshalb im Haushaltsrechner zwar aus Vorschlägen generiert, aber dort nicht
weiter verarbeitet werden. Letzteres obliegt einer
\emph{Bürgereinwandsbearbeitungsstelle} (BEBS), die von der Stadt eingerichtet
werden muss.

Vor der Implementierung weitergehender Funktionen muss deshalb in einem
Beteiligungskonzept festgelegt werden, welche Funktion ein Interaktiver
Haushaltsrechner erfüllen soll und darf. In der Besprechung vom 06.05.2015
wurde von Seiten der Stadtverwaltung betont, dass die Beantwortung jedes
Bürgereinwand vorgeschrieben sei und insofern ausreichend Personalmittel zur
Verfügung stehen müssen. Die Online-Einbindung von Bürgereinwänden habe für
die Stadt Leipzig hohe Priorität, derzeit sei jedoch rechtlich noch einiges zu
klären. Fraglich sei u.a., wie mit „Like-Funktionen“ bzw. „Mitzeichnungen“
umzugehen sei -- handelt es sich dann um jeweils eigenständige Einwände -- und
ob jeder Nutzer individuell zu informieren sei oder eine Antwort in Form eines
Beitrags ausreichen würde. Darüber hinaus müsse geprüft werden, ob jeder
Bürgereinwand (ohne personenbezogene Daten) sofort frei online einsehbar sein
kann oder zuvor eine BEBS einbezogen werden muss.

\subsubsection*{Stand der Umsetzung} 

Mit Blick auf diese bis zuletzt unklare Anforderungslage sowie die generellen
Unwägbarkeiten, ob überhaupt ein Forum wie im Prototyp vorgeschlagen zum
Einsatz kommt, wurde deshalb nur die folgende Variante eines \emph{einfachen
  Bürgereinwands} umgesetzt:
\begin{itemize}
\item Angemeldete Benutzer können einzelne Vorschläge über einen Knopf als
  (einfachen) Bürgereinwand einreichen. 
  
  Ein spezifischer Registrierungsprozess, wie für \emph{qualifizierte
    Bürgereinwände} gefordert, findet dabei nicht statt. Moderatoren haben
  Zugriff auf die für sie durch die Administration freigeschalteten
  persönlichen Daten der Einreicher, welche diese zu ihrem Account übermittelt
  haben, und können auf diesem Weg Benutzerinformationen einsehen und
  ggf. weitere Informationen über die Kontaktfunktionen der Plattform
  anfordern. 

\item So markierte Vorschläge werden zusammen mit dem Verweis auf den
  Einreicher in einer drupalinternen Liste aggregiert, die nur in der
  Moderatorensicht zugänglich ist.

  Damit kann erfasst werden, welche Vorschläge wie oft als Bürgereinwände
  eingereicht wurden.

\item Reaktionen der BEBS auf einzelne Vorschläge (Verwaltungsstandpunkt)
  können als Beitrag im entsprechenden Vorschlagsthread veröffentlicht werden.
\end{itemize}

\subsubsection*{Mögliche Erweiterung des Konzepts} 

Dieses Konzept könnte wie folgt zu einem \textbf{Konzept zum Einreichen von
  qualifizierten Bürgereinwänden} erweitert werden:
\begin{itemize}
\item Über den Knopf „Bürgereinwand einreichen“ wird aus einem
  Vorschlags-Thread heraus ein Webformular generiert, über das der
  qualifizierte Bürgereinwand (QBE) erstellt werden kann.  Jeder QBE hat damit
  eine QBE-ID, einen Ersteller, ein Erstellungsdatum, eine Thread-ID und eine
  Produktbereichs-ID als Metainformationen.
\item QBE werden in einer separaten Datenstruktur gesammelt und können über
  eine gesicherte Webschnittstelle der BEBS ausgeliefert werden.
  Weiterführende Rollen und Prozessabläufe sind zum produktiven Einsatz des
  Prototyps noch zu definieren und die Datenweitergabe entsprechend technisch
  umzusetzen.  Dabei sind Fragen des Datenschutzes zu berücksichtigen.
\item QBE können damit in der BEBS auf der Ebene von Threads aggregiert und im
  Bündel ausgewertet werden.  Die Thread-ID erlaubt es auch, seitens der BEBS
  einen Kommentar im jeweiligen Thread mit dem Verwaltungsstandpunkt zu
  ver"|öffent"|lichen. 
\end{itemize}
Damit wären auch die Anforderungen umgesetzt, die am 08.06.2015 noch einmal wie
folgt konkretisiert wurden:
\begin{itemize}
\item Einwände formulieren können und Rückinfo zum Einwand erhalten. Dabei ist
  die Eingabe weiterer persönlicher Daten erforderlich. Bei
  datenschutzrelevanten Eingaben müssen Seiten durch ein SSL-Zertifikat
  gesichert werden. 
\item Dies soll möglichst mit einem Formular gelöst werden. Dazu muss eine
  Adresse aus dem Dezernat II zur Bearbeitung hinterlegt werden.
\item Dabei muss der Unterschied zwischen Kommentar und „förmlich rechtlicher
  Einwand” berücksichtigt werden, siehe Formular für Online-Petitionen.
\end{itemize}

\subsubsection*{Weitere Anforderungen} 

Weitere Aspekte -- wie das Einpflegen von Bürgereinwänden, die über andere
Kanäle eingereicht werden, in die Plattform durch die BEBS -- müssten im
Rahmen der Erarbeitung eines BEBS-Betriebskonzepts entschieden, prozessual
untersetzt und ggf.\ über eine Plattformerweiterung technisch angebunden
werden.  Im aktuellen Konzept kann die BEBS hierzu eigene Threads eröffnen,
einen Bürgereinwand „im Namen des Bürgers“ auf dem beschriebenen Weg in die
Plattform einspielen und ggf.\ auch beantworten. 

In diesem Zusammenhang sind die folgenden weiteren Anforderungen zu
berücksichtigen: 
\begin{itemize}
\item Ein Benutzer kann sich einem Bürgereinwand anschließen, vergleichbar dem
  Mitzeichnen einer Petition.
\item Grundsätzlich sei die Stadt mit den bisherigen Angeboten im Rahmen der
  vorhandenen Möglichkeiten nicht schlecht aufgestellt. Problematisch sei
  allerdings, dass Beteiligung im Prozess grundsätzlich erst nach der
  Veröffentlichung des Haushaltsentwurfes möglich sei, aber zu diesem
  Zeitpunkt viele Weichen schon gestellt sind.
\item Zudem sollte der Umgang mit Einwänden und Vorschlägen möglichst noch
  transparenter gestaltet werden.
\item Alle eingehenden Vorschläge (Brief, Bürgerworkshop, E-Mail,
  Online-Diskussionsforen, usw.) müssten idealerweise frei einsehbar seien.
\end{itemize}

\section{Allgemeine Anforderungen} 

\subsection{Barrierefreiheit} 

\subsection*{Vorgaben für die Barrierefreiheit}

Die Vorgaben sind in der BITV 2.0 Verordnung geregelt.  Als barrierearm gilt
eine Website, wenn 90 von 100 Punkten erreicht werden.  Dafür gibt es Tests
(Selbsttest oder
Firmen)\footnote{\url{http://www.bitvtest.de/bitvtest/das_testverfahren_im_detail/pruefschritte.html}}.

\subsection*{Realisierung im Prototyp}

Der Prototyp wurde auf der Basis von Drupal erstellt.  Die geforderten
BITV-Werte werden bei professionellen Drupal-Instanzen
erreicht\footnote{\url{http://www.bitvtest.de/90plus/cms/drupal.html}}.
Drupal und der Prototyp setzen dabei auf dem diesbezüglich optimierten
Bootstrap-CSS-Framework\footnote{\url{http://getbootstrap.com}} auf.

\begin{itemize}\itemsep0pt
\item Die Überprüfung des Prototyps mit einem automatisierten Werkzeug nach
  amerikanischem Standard hat dieser erfolgreich bestanden.
\item Für deutsche Standards haben wir keine automatisierten kostenfreien
  Selbsttests gefunden, nur einen Fragebogentest, der jedoch
  ressourcenintensiv ist.
\item Die Stadt verwendet ein eigenes CSS-Framework. Im Projekt wurde nur eine
  Anpassung an das Look-and-Feel der Stadtseiten auf der Basis des
  Bootstrap-CSS-Frameworks ausgeführt.  Der zukünftige Betreiber hat zu
  entscheiden, ob dies so übernommen wird oder noch einmal eigene Ressourcen
  zur Design-Anpassung eingesetzt werden.  Im zweiten Fall steht die Bewertung
  auf der Basis der BITV 2.0 sowieso neu.
\end{itemize}
Mit Blick auf diese Unwägbarkeiten wurden keine Projektressourcen eingesetzt,
um über die durch das Bootstrap-CSS-Framework in diesem Bereich bereits
garantierten hohen Standards hinauszugehen.

\subsection{Integration in die Stadthomepage}

\subsection*{Rahmenbedingungen}

Der Prototyp wurde auf einem eigenen Webserver entwickelt und erprobt.  Zur
Portierung auf einen anderen Webserver wurde ein detailliertes
Relokationskonzept entwickelt, das in der „Installations- und
Konfigurationsbeschreibung“ genauer ausgeführt ist. 

Der Prototyp wurde im Rahmen der Möglichkeiten, die Drupal und das
Bootstrap-CSS-Framework dafür bieten, an das Look-and-Feel der Stadtseiten
angepasst. Dies wurde im Gespräch am 06.05.2015 mit der Online-Redaktion der
Stadt umfassend diskutiert und im Nachgang dafür eine HTML-Vorlage der Stadt
zur Verfügung gestellt.

Damit sind die Voraussetzungen geschaffen, dass der Prototyp in der aktuellen
Ausprägung nach einer positiven Einsatz-Entscheidung durch den künftigen
Betreiber auf einem eigenen Webserver ausgerollt, in Betrieb genommen und über
Verlinkungen leicht in die Stadthomepage integriert werden kann. 

Eine tiefere Integration in die IT-Struktur der Stadthomepage ist grundsätzlich
möglich, ließe sich jedoch nur mit einem erheblichen Mehraufwand und
umfassender Einbeziehung der städtischen IT-Verantwortlichen realisieren.
Überlegungen hierzu sind im Abschnitt 1.2 „Weiterführende Arbeiten“ genauer
ausgeführt.

\subsection*{Anforderungen}

Folgende Punkte ergaben sich aus der Diskussion mit der Online-Redaktion der
Stadt:
\begin{itemize}\itemsep0pt
\item Einhaltung der Anforderungen nach BITV 2.0 durch den Prototyp
  \kommentar{Da dies vom eingesetzten CSS-Framework abhängt, wurde diese Frage
    über erste Untersuchungen hinaus nicht weiter verfolgt, siehe dazu den
    Abschnitt „Allgemeine Anforderungen“.}
\item Valides XHTML des Prototyps. 75 Fehler bei Validierung der URI
  \begin{center}
    \url{http://pcai042.informatik.uni-leipzig.de/~swp15-ihr/Drupal/}
  \end{center}
  mit \url{https://validator.w3.org} 

\kommentar{Nach der konsequenten Umstellung auf HTML 5 liefert der Test nun
  keine Validierungsfehler mehr -- nur eine Warnung. }
\item Validierung CSS nicht erfolgreich: 591 Fehler bei Validierung der URI
  \begin{center}
    \url{http://pcai042.informatik.uni-leipzig.de/~swp15-ihr/Drupal/}
  \end{center}
  mit \url{http://www.css-validator.org/}. 

\kommentar{Das eingesetzte CSS \texttt{Drupal/themes/IHR/css/style.css} ist
  eine (minimale) Modifikation des Bootstrap-CSS, eines der weltweit
  meistverwendeten CSS. Die 199 Fehler, die eine Analyse mit
  \url{http://jigsaw.w3.org/css-validator/} liefert, hängen zum großen Teil
  damit zusammen, dass Bootstrap Elemente im Draft-Status verwendet, die alle
  modernen Browser beherrschen, die aber noch nicht vom Validator erkannt
  werden.  Das „Parse Error [empty string]“ ist sogar ein Bug des
  Validators\footnote{Siehe
    \url{http://w3.org/Bugs/Public/show_bug.cgi?id=11975}.}.

  Offizielle Aussage des Bootstrap-Entwicklungsteams: „Generally speaking, we
  don't worry about W3C validation. Practically speaking, it honestly doesn't
  make much sense to strive for it in most production environments if industry
  accepted practices are viewed as errors.“\footnote{Siehe
    \url{https://github.com/twbs/bootstrap/issues/6398#issuecomment-11692542}.}
  }

\item Verhalten bei deaktiviertem Javascript
  \begin{itemize}
  \item Alle Seiten können auch mit deaktiviertem Javascript aufgerufen
    werden. 
  \item Einnahmen und Ausgaben sind bei deaktiviertem Javascript nicht
    verfügbar (Kerninhalt)
  \item Suchvorschläge sind nicht verfügbar.
  \end{itemize}

\kommentar{Unser Prototyp verwendet wie alle modernen Webapplikationen
  Javascript an verschiedenen Stellen, insbesondere zur visuellen Darstellung,
  und kann deshalb nicht sinnvoll mit deaktiviertem Javascript betrieben
  werden. }

\item Look and Feel der Stadtseiten in den Prototyp einbauen.
\kommentar{Umgesetzt im Drupal-Thema des Prototyps.}
\item Angemessene Druckversion für die Webseiten des Prototyps.
\kommentar{Umgesetzt im Drupal-Thema des Prototyps.}
\item Anbindung des Drupal Statistics Counter an das Webanalysesystem der
  Stadt
\kommentar{Nicht untersucht.}
\item Responsive Design: Darstellung soll sich auf die verschiedenen
  Bildschirmgrößen anpassen, ebenso bei den Mobilgeräten. 
\kommentar{Umgesetzt im Drupal-Thema des Prototyps als eine Grundfunktionalität
  des Bootstrap-CSS-Frameworks.}

\item Unterstützung verschiedener Browserversionen.
  \begin{itemize}
  \item Die Online-Redaktion stellt fest: Die Darstellung mit IE9, Firefox
    31.7, Opera 23, Safari 5.1.7 und Chrome 41 sind identisch.
  \end{itemize}
\item Überschriften müssen korrekt mit den HTML-Strukturelementen h1 bis h6
  ausgezeichnet sein und die Inhalte der Seite erschließen. (H1 fehlt auf
  allen Seiten und es werden Ebenen übersprungen.)

\kommentar{Drupal verwendet intern unsichtbare Überschriften der Ebenen h1 und
  h2 zur Markierung des Seitenaufbaus, so dass nach entsprechender
  Überarbeitung sichtbare eigene Überschriften nun sämtlich auf der Ebene h3
  beginnen. }
\end{itemize}
Offene Fragen:
\begin{itemize}
\item Beschriftungen (label-Elemente) sollen über das entsprechende Markup
  (das for-Attri"|but) mit den Eingabefeldern verknüpft sein, zu denen sie
  gehören. (fehlt bei den Such-Formularen und im Forum.)
\item Interaktive Bedienelemente wie Links und Schaltflächen haben
  programmatisch ermittelbare Namen und Rollen (ist bei den Suche-Buttons
  nicht erfüllt).
\item Kontrast nicht ausreichend bei Button „Neues Benutzerkonto erstellen“.
\item Verwendete englische Sprache im Forum ist nicht ausgezeichnet (für
  Screenreader problematisch).
\end{itemize}

\section{Datenmodell}

Für Details zum Datenmodell wird auf das Dokument „Designprinzipien des RDF
Data Stores“ verwiesen.

Mit dem URI-Präfix \url{http://haushaltsrechner.leipzig.de/Data} sind die
Voraussetzungen geschaffen, dass die RDF-Datenbasis des neuen
Haushaltsplanrechners von einem Betreiber in der Domäne \texttt{leipzig.de}
auch nach Linked Open Data Prinzipien ausgeliefert werden kann.

\subsection{Darstellung finanzieller Spielräume der Kommune}

Anforderungen:
\begin{itemize}\itemsep0pt
\item Auswertung der Informationen zu finanziellen Spielräumen der einzelnen
  Produktnummern und Einbau dieser Informationen in das visuelle Konzept des
  Informationsteils. 
\item Tatsächliche Stellschrauben und Einflussmöglichkeiten im städtischen
  Haushalt für den interessierten Bürger müssen klarer erkennbar sein, als
  beim ehemaligen „Haushaltsplanrechner“. Einflussgrößen sind erarbeitet
  worden: der Vorschlag soll so umgesetzt werden, ggf. kann im Nachhinein noch
  die eine oder andere Größe angepasst werden.
\end{itemize}

Diese Informationen sind im Produktmodell auf der Ebene der Schlüsselprodukte
aus den vom Dezernat für Finanzen der Stadt Leipzig zur Verfügung gestellten
Spielraumangaben als reelle Zahl $g$ mit $1\le g\le 3$ interpoliert, da die
Angaben teilweise nur auf der Ebene von Unterprodukten vorlagen und in die
Produktbeschreibungen der einzelnen Schlüsselprodukte integriert.

Eine darüber hinausgehende Darstellung von Spielräumen lässt sich im
Produktmodell leicht nachrüsten, wenn man sich über einen
Berechnungsalgorithmus von Spielraumangaben für Knoten des Produktgraphen aus
den Spielraumangaben der Kindknoten geeinigt hat. 
\pagebreak

\subsection{Vergleiche zwischen verschiedenen Jahren} 

Anforderung:
\begin{itemize}\itemsep0pt
\item Darstellung der Entwicklung einzelner Produktgruppen im zeitlichen
  Verlauf über mehrere Jahre und prozentuale Veränderung. 
\end{itemize}

Mit dem Prototyp wird eine Serie von Jahresdaten der Planung des
Ergebnishaushalts für die Jahre 2014 bis 2019 ausgeliefert, die aus Daten
extrahiert wurden, die uns das Dezernat für Finanzen der Stadt Leipzig zur
Verfügung gestellt hat.  Auf dieser Basis wird prototypisch demonstriert, wie
Daten verschiedener Jahre inspiziert und auch vergleichend dargestellt werden
können.

Diese Plandaten sind allerdings nur geeignet, die Möglichkeiten
\emph{prinzipiell} zu demonstrieren, da sie von den veröffentlichten
Haushaltsplandaten deutlich abweichen.  Für einen Produktivbetrieb müssten
entsprechende konsolidierte Datenreihen zur Verfügung gestellt werden. 

\end{document}
