\documentclass[a4paper,11pt,twoside]{article}
% Encoding lokal, da manche Editoren danach schauen
\usepackage[utf8]{inputenc} 
\usepackage{ihr-15}

\title{Designprinzipien des IHR-15 RDF Data Stores}
 
\date{Version vom 21. August 2015}

\begin{document}
\maketitle
\tableofcontents
\thispagestyle{empty}
\newpage
\seitezwei
\newpage

\section{Allgemeines}

Mit dem Übergang zum „Doppik“ haben sich 2012 auch in Leipzig die Grundsätze
der Haushaltsführung geändert.  Mit \emph{Ergebnishaushalt} und
\emph{Finanzhaushalt} unterscheidet die „doppelte Buchführung“ zwischen der
Darstellung von Einnahmen und Ausgaben (Ergebnishaushalt) sowie von
Aufwendungen und Ergebnissen (Finanzhaushalt).  Die Darstellung im
veröffentlichten Haushalt der Stadt Leipzig (4 Bände pro Jahr, die als PDF
online bei \url{http://leipzig.de} eingesehen werden können) nach
verschiedenen Systematiken (Produktgruppen und Produkte, Finanzpositionen,
Kostenstellen und Kostenarten) macht die Sache zusätzlich unübersichtlich.

Weiterhin sind in einem einzelnen Haushaltsplan nicht nur die Daten für das
aktuelle Jahr berücksichtigt, sondern auch Prognosen für die Folgejahre
ausgewiesen.  Schließlich ist noch zu unterscheiden zwischen zu verschiedenen
Zeitpunkten fixierten \emph{Plandaten} und dem erst retrospektiv verfügbaren
\emph{Haushaltsergebnis}, so dass selbst für ein Jahr verschiedene
Zahlenmaterialien im Umlauf sind.

\begin{quote}
  Bis zum Projektende konnten fundamentale Unstimmigkeiten in der
  Interpretation der durch die Kämmerei zur Verfügung gestellten Haushaltsdaten
  nicht ausgeräumt werden, so dass die mit dem Prototyp ausgelieferten Daten
  \textbf{nur kursorisch die Funktionalitäten des Haushaltsplanrechners
    demonstrieren, nicht aber belastbare Informationen zur Haushaltssituation
    der Stadt Leipzig wiedergeben.} 
\end{quote}


Im folgenden Text wird ein Grundverständnis der kommunalen Haushaltssystematik
vorausgesetzt.

\section{Die Datenlage}

Dem Projekt standen initial (seit Oktober 2014) seitens die Kämmerei
\begin{itemize}
\item[(1)] \emph{Haushaltsplandaten zum Jahr 2014}
\end{itemize}
zur Verfügung, welche in einer Datei \texttt{Ratsinfo} die Haushaltssystematik
in einer zusammenfassenden Darstellung bis zur Ebene 4 der
\emph{Bezüge}\footnote{So lautet die für Ergebnishaushalt und Finanzhaushalt
  gemeinsame Spaltenüberschrift in der Datenquelle.}  sowohl des
Ergebnishaushalts als auch des Finanzhaushalts abbildeten sowie in einer Datei
\texttt{ErgHH\_PB11\_2014} beispielhaft detailliertere Daten für einen
Produktbereich (PB 11) des Ergebnishaushalts sowie in einer Datei
\texttt{FinHH} Daten zum Finanzhaushalt enthielten. In den Haushaltsplandaten
waren Planansätze für die Jahre 2013 bis 2017 enthalten.
\begin{quote}
  Problematisch war von Anfang an die Interpretation der uns übergebenen Daten.
  Kumuliert über alle 1128 Posten des Produktbereichs 11 ergaben sich etwa
  folgende Summen für Einnahmen und Ausgaben im Planansatz 2014:
  \begin{center}
    \begin{tabular}{l|r|r}
      & \texttt{ErgHH\_PB11\_2014}& \texttt{Ratsinfo}\\\hline
      Einnahmen & 976\,928\,992 & 84\,642\,290 \\
      Ausgaben & -938\,157\,157 & -620\,150
    \end{tabular}
  \end{center}
Im Weiteren wurde mit den Daten aus der Datei \texttt{Ratsinfo} gearbeitet. 
\end{quote}

An diesen Daten ließ sich im Teil \emph{Ergebnishaushalt} die Systematik der
Produktbezeichnungen rekonstruieren (ein Bezug pro Zeile, jede Zeile ist mit
Angaben zu Produktbereich, -gruppe und -untergruppe versehen).  Diese
4-Ebenen-Systematik wird im folgenden als \textbf{Produktgraph} bezeichnet, die
genauere Charakterisierung der einzelnen Knoten dieses Graphen als
\textbf{Produktmodell}. 

Am 27. Juli 2015 wurde in einem Gespräch mit der Kämmerei klar, dass die in der
Systematik angegebenen Produkte nicht gleichwertig sind, sondern Produkte mit
10-stelligen Nummern die Rolle von \emph{Schlüsselprodukten} spielen, zu denen
auch zum großen Teil ausführlichere Produktsteckbriefe existieren, während
andere Produkte als Unterprodukte anzusehen sind. Auf dieser Basis wurden in
einem \emph{Redesign des Produktmodells} allein die Schlüsselprodukte als
vierte Ebene des Produktgraphen belassen, diese aber begriff"|lich von
PSP-Elementen getrennt\footnote{Zu einer 10-stelligen Produktnummer gibt es nun
  sowohl ein PSP-Element als auch ein Schlüsselprodukt.}. Das PSP-Element eines
einzelnen Datensatzes der Primärdaten wird eindeutig einem Schlüsselprodukt
zugeordnet und damit nun Einnahmen und Ausgaben bereits auf dieser Ebene
kumuliert.

Eine solche Systematik konnte aus den Angaben im Teil \emph{Finanzhaushalt}
nicht rekonstruiert werden, da hier Bezüge in mehreren Zeilen vorkamen und in
den meisten Fällen auch die Einordnung in den Produktgraphen fehlte.

Der Entwicklung unseres Haushaltsrechners wurden deshalb die Plandaten des
Ergebnishaushalts aus dem Jahr 2014 zu Grunde gelegt.

Im Mai 2015 wurde uns durch die Kämmerei
\begin{itemize}
\item [(2)] ein \emph{Produktplan 2015 mit Spielraumangaben} 
\end{itemize}
und im Juni 2015 
\begin{itemize}
\item [(3)] eine \emph{Übersicht mit 205 Produktsteckbriefen} 
\end{itemize}
übergeben, aus denen sich weitere Informationen zur Systematik des
Ergebnishaushalts entnehmen ließen: 
\begin{itemize}
\item Im \emph{Produktplan} sind in jeder Zeile entweder ein PSP-Element mit
  einer Bezeichnung \texttt{1.100.*} oder ein „Innenauftrag“ dargestellt. Die
  PSP-Nummern stimmen mit dem überein, was in der Ratsinfo-Darstellung im
  Bereich \emph{Ergebnishaushalt} als \emph{Bezug} bezeichnet ist.
\item Unter Weglassung der Punkte ergeben sich mindestens 10-stellige
  PSP-Nummern, wobei sich die 10-stelligen PSP-Nummern zum überwiegenden Teil
  auch in den Produktsteckbriefen wiederfinden.  Der Präfix 1100 weist darauf
  hin, dass es sich um ein Produkt aus dem Ergebnishaushalt handelt. 
\item In einem Gespräch am 27. Juli 2015 wurde uns der Hintergrund erläutert --
  die 10-stelligen PSP sind \emph{Schlüsselprodukte}, weitere PSP werden als
  \emph{Unterprodukte} bezeichnet und haben eine Nummer, aus der sich die
  Nummer des zugehörigen Schlüsselprodukts als Präfix ergibt. Unterprodukte
  haben im Gegensatz zu Schlüsselprodukten keine ausführ"|liche Beschreibung. 
\item Aus diesen PSP-Nummern lässt sich die Einordnung in den Produktgraphen
  unmittelbar ablesen. 
\end{itemize}
Auf dieser Basis wurden die RDF-Graphen \textbf{NeuesProduktModell}
(Produktmodell des Ergebnishaushalts), \textbf{NeuerProduktgraph} und
\textbf{Produktsteckbriefe} im Rahmen eines Redesigns neu erstellt, in denen
die Systematik des Ergebnishaushalts für die Zwecke des HH-Rechners umfassend
dargestellt ist.  Diese Darstellung ist mit dem Produktplan
(S. 686\,ff. Haushaltsplan, Band 1) abgeglichen.

Im Mai 2015 wurden uns seitens der Kämmerei 
\begin{itemize}
\item [(4)] \emph{detaillierte Daten zum Haushaltsplan 2015/16} übergeben: 
  \begin{itemize}
  \item \texttt{ErgHH\_1} -- Darstellung des \emph{Ergebnishaushalts} in
    ähnlicher Systematik wie die Datei \texttt{ErgHH\_PB11\_2014},
  \item \texttt{FinHH\_lfd\_Verwaltungsteaetigkeit} -- Darstellung des
    \emph{Finanzhaushalts laufende Verwaltungstätigkeit} sowie 
  \item \texttt{invest\_F8} -- Darstellung des \emph{Finanzhaushalts} in
    ähnlicher Systematik wie \texttt{FinHH}.
  \end{itemize}
\end{itemize}
Es fehlte eine kumulierte Übersicht \texttt{Ratsinfo}, so dass sich die
bisherige Datentransformation auf die neuen Daten so nicht anwenden ließ. 

In den folgenden Monaten wurde eine Datentransformation auf der Basis des
vorgelegten Ergebnishaushalts\texttt{ErgHH\_1} neu aufgebaut, welche die
bisherigen Erfahrungen berücksichtigt.  Die Datensätze dieses
Ergebnishaushalts sind als „Rechnungsendbetrag 2013“ oder „Planwert 2014“ bis
„Planwert 2018“ ausgewiesen, so dass sich zunächst einmal aus einer
csv-Version der Primärquelle entsprechende Datensätze für die einzelnen Jahre
für die weiteren Transformationen extrahieren lassen.

Diese Datenbasis wurde noch einmal Ende Juli 2015 aktualisiert und uns durch
die Kämmerei auch die Daten zum Haushalt 2016 sowie eine kumulierte Übersicht
\texttt{Ratsinfo} (allerdings mit fehlerhafter Zeichensatz-Kodierung) zur
Verfügung gestellt.  Diese Planwertdaten für die Ergebnishaushalte der
Jahre 2014 bis 2019 bildeten die Basis für die weiteren Datentransformationen. 
\begin{quote}
  Auch in diesen Daten gibt es erhebliche Differenzen.  Kumuliert über alle
  18\,848 Posten des aus der Primärquelle \texttt{ErgHH\_1} extrahierten
  detaillierten Ergebnishaushalts bzw.\ über alle 1\,158 Posten des
  Ergebnishaushalts der Datei \texttt{Ratsinfo\_2015} ergaben sich folgende
  Summen für Einnahmen und Ausgaben im Planansatz 2015:
  \begin{center}
    \begin{tabular}{l|r|r}
      & \texttt{ErgHH\_2015}& \texttt{Ratsinfo}\\\hline
      Einnahmen & -3\,399\,306\,619,31 & -1\,487\,889\,183,26 \\
      Ausgaben & 3\,384\,184\,302,47 & 1\,472\,766\,866,42
    \end{tabular}
  \end{center}
Da allerdings in der Datei \texttt{Ratsinfo\_2015} eine detaillierte
Gegenüberstellung von Einnahmen und Ausgaben nur für 2015 gegeben wird, für
Vergleichsjahre aber nur eine konsumierte Sicht dargestellt ist, im
detaillierten Ergebnishaushalt dagegen in den Planansätzen für die Jahre 2014
bis 2019 Einnahmen und Ausgaben gegenübergestellt werden, wurde das Redesign
auf der Basis dieser detaillierten Daten ausgeführt, um die Möglichkeiten der
visuellen und tabellarischen Gegenüberstellung von Haushaltsdaten verschiedener
Jahre auszuloten.
\end{quote}
Im Gespräch am 27. Juli 2015 wurde auch besprochen, dass für den HH-Rechner die
Einnahmen und Ausgaben auf der Ebene der Schlüsselprodukte aggregiert werden
sollen, da nur für diese Produktsteckbriefe vorliegen und damit eine
ausreichende Informationsbasis für Erläuterungen zur Verfügung steht.  

Für eine ähnlich systematische Transformation der Finanzhaushalte in eine
Datenbasis für den HH-Rechner bleiben nach wie vor viele Fragen offen.

\section{RDF Data Cube -- Konzeptionelle Vorbemerkungen}

\subsection{Die RDF Data Cube Systematik}

Im weiteren Text wird davon ausgegangen, dass die grundlegenden Konzepte des
RDF Data Cube Ansatzes \cite{RDF-Cube} bekannt sind.  \texttt{qb:} steht wie
üblich für den RDF Cube Namensraum.

Im weiteren Text wird an verschiedenen Stellen im Kontext des RDF Data Cube
Konzepts immer wieder von RDF-Klassen und deren Instanzen die Rede sein.  Ich
benutze dabei für Instanzen der RDF-Klasse \texttt{qb:WasAuchImmer}
durchgängig Bezeichnungen wie \emph{ein WasAuch\-Immer}.

\subsection{RDF Data Cubes}

Die Datenbasis des HH-Rechners ist in einzelne \emph{RDF Data Cubes} als
Datasets aufgeteilt, in denen jeweils alle Datensätze (Observations) aus einer
Quelle und zu einem Jahr zusammen mit dem Verweis auf die formale Beschreibung
der Struktur dieser Observations zusammengefasst sind.

Ein RDF Data Cube enthält \emph{ein DataSet} (eine Instanz der RDF-Klasse
\texttt{qb:DataSet}) sowie gleichartig strukturierte \emph{Observations}
(Instanzen vom RDF-Typ \texttt{qb:Observation}), die alle einer gemeinsamen
\emph{Modellstruktur} für die Datenerfassung folgen, deren Beschreibung sowohl
syntaktisch als auch semantisch als eine \emph{DataStructureDefinition} über
das Prädikat \texttt{qb:structure} des DataSets referenziert ist.  Im folgenden
Code-Beispiel ist der Zusammenhang zwischen einer Observation
\texttt{ihrdata:EH\_15G\_Plan14-Bezug1100111101} und dem Dataset
\texttt{ihrds:EH\_15G\_Plan14} aus einem unserer Cubes prototypisch
dargestellt.
\begin{code}
ihrds:EH\_15G\_Plan14 a qb:DataSet ;\+\\
  rdfs:label {\dq}Haushaltsdaten 2015 der Stadt Leipzig ...{\dq}@de ;\\
  rdfs:comment {\dq}...{\dq}@de ;\\
  dct:source {\dq}Kämmerei der Stadtverwaltung Leipzig{\dq} ;\\
  qb:structure ihr:DSDShort .\-\\[6pt]
ihrdata:EH\_15G\_Plan14-Bezug1100111101 a qb:Observation ; ... \+\\
  qb:dataSet ihrds:EH\_15G\_Plan14 . 
\end{code}

Um statistische Vergleichbarkeit zu erreichen, folgt eine zusammengehörende
Serie von solchen \emph{DataSets} derselben Modellstruktur, also derselben
\emph{DataStructureDefinition}, die im Fall unseres HH-Rechners die URI
\texttt{ihr:DSDShort} hat und in der Datei \texttt{HaushaltLeipzigCube.ttl}
definiert ist, die wir uns nun genauer anschauen wollen.

Eine \emph{DataStructureDefinition} beschreibt ein RDF-Schema, das seinerseits
das RDF-Schema der Observations beschreibt. Die Beschreibung dieser
Beschreibung (Metamodellebene) setzt dabei konsequent auf den (Meta)-Konzepten
von RDF-Modellen \cite{RDFS} wie \texttt{rdfs:subClassOf} und
\texttt{rdfs:subPropertyOf} auf.  Eine \emph{DataStructureDefinition} besteht
aus einer Menge von Instanzen vom RDF-Typ \texttt{qb:ComponentSpecification},
die der \emph{DataStructureDefinition} über das Prädikat \texttt{qb:component}
zugeordnet sind und mit denen die in den Observations verwendeten
\emph{Prädikate} genauer beschrieben werden. Hierbei wird ein besonderes
syntaktisches Moment von RDF ausgenutzt -- Prädikate einer Ebene können auf
einer Metaebene selbst \emph{Subjekt} von Beschreibungen sein. 

\subsection{ComponentSpecification und ComponentProperties}

Um solche Prädikate zu verschiedenen Prädikatklassen zusammenzufassen, wird ein
weiteres Indirektionsprinzip angewendet, mit dem Vererbungsstrukturen in RDF
modelliert werden können: Über das RDF-Oberprädikat
\texttt{qb:componentProperty} ist einer \emph{ComponentSpecification} eine
Instanz vom RDF-Obertyp \texttt{qb:ComponentProperty} zugeordnet, von denen es
verschiedene Unterprädikate mit zugehörigen Unterklassen als Range gibt:
\begin{itemize}\itemsep0pt
\item \texttt{qb:dimension} mit Range \texttt{qb:DimensionProperty},
\item \texttt{qb:attribute} mit Range \texttt{qb:AttributeProperty},
\item \texttt{qb:measure} mit Range \texttt{qb:MeasureProperty}.
\end{itemize}
Im folgenden Beispiel ist in der DataStructureDefinition \texttt{ihr:DSDShort}
eine ComponentSpecification \texttt{ihr:jahrComponent} definiert, die
ihrerseits eine spezielle DimensionProperty \texttt{ihr:jahr} definiert, die
den Wertebereich (Range) \texttt{xsd:gYear} hat und als Prädikat in
Observations verwendet werden kann.
\begin{code}
ihr:DSDShort a qb:DataStructureDefinition ;\+\\
  rdfs:label {\dq}Data Structure Definition für den Leipziger Haushalt{\dq}@de ;\\
  qb:component ihr:jahrComponent .\-\\[6pt]
ihr:jahrComponent a qb:ComponentSpecification ;\+\\
  rdfs:label {\dq}Jahr-Component{\dq}@de ;\\
  qb:dimension ihr:jahr .\-\\[6pt]
ihr:jahr a rdf:Property, qb:DimensionProperty ;\+\\
  rdfs:label {\dq}Jahr{\dq}@de ;\\
  rdfs:range xsd:gYear .
\end{code}
Jeder \emph{ComponentSpecification} ist auf diese Weise eine
\emph{ComponentProperty} zugeordnet, die zu einer der drei definierten
Unterklassen gehören kann, was daran zu erkennen ist, welches der Prädikate
\texttt{qb:dimension}, \texttt{qb:attribute} oder \texttt{qb:measure} in der
Definition verwendet wird.  

Diese auf den ersten Blick etwas umständliche Indirektion macht sich
erforderlich, weil RDF nur 3-Wort-Sätze kennt.  Explizite URIs für
\emph{ComponentSpecifications} können durch „blank nodes“ umgangen werden,
allerdings folgen wir der Empfehlung, „blank nodes“ aus anderen syntaktischen
Gründen konsequent zu vermeiden.  

Wichtig für das weitere Verständnis der Struktur der Observations sind allein
die \emph{ComponentProperties} in ihren drei Typ-Ausprägungen als
\emph{DimensionProperties}, \emph{AttributeProperties} und
\emph{MeasureProperties}.  Jede \emph{ComponentProperty} ist insbesondere eine
\texttt{rdf:Property} mit den Prädikaten \texttt{rdfs:label},
\texttt{rdfs:comment} und \texttt{rdfs:range}.

Als \texttt{rdfs:range} können standardmäßige RDF-Typbezeichner-Klassen
verwendet werden.  Oft müssen aber für spezielle Modelle weitere Klassen
definiert werden, insbesondere als Wertebereich für \emph{DimensionProperties}.
Für derartige Definitionen wurde das Konzept des \texttt{qb:mea\-sure\-Type}
entwickelt, um einen generischen Rahmen für entsprechende Typklassen zu
entwickeln.  Dieses Konzept kommt im IHR-Cube derzeit nicht zum Einsatz.

Für die eingeführten Properties sind neben syntaktischen auch semantische
Aspekte zu definieren. Dies erfolgt mit dem Prädikat \texttt{qb:concept}, über
welches jeder ComponentProperty (und damit auch jeder der drei möglichen
UnterProperties) eine formal definierte Semantik aus geeigneten Konzeptwerken
zugeordnet werden kann.  Im IHR-Modell wird hierfür konsequent das im
Statistikbereich inzwischen etablierte SDMX-Vokabular \cite{SDMX} referenziert.

\section{Zur Struktur unserer Daten}

\subsection{Die Struktur unserer RDF Data Cubes}

Jeder unserer Daten-Cubes enthält eine Menge von Observations (jeweils aus
einer spezifizierten Primärquelle zu einem Jahr) zusammen mit der zugehörigen
DataSet-Definition und einer Beschreibung des RDF-Graphen als Instanz von
\texttt{owl:Ontology}.  Dazu sind zwei Namensraum-Präfixe
\begin{itemize}
\item \texttt{ihrdata: <http://haushaltsrechner.leipzig.de/Data/Observation/>}
  und 
\item \texttt{ihrds: <http://haushaltsrechner.leipzig.de/Data/Dataset/>}
\end{itemize}
definiert, die als Präfixe für URIs der Observations (\texttt{ihrdata})
bzw.\ des Datasets (\texttt{ihrds}) dienen.  

Die Bezeichnungen des RDF Graphen, des dort enthaltenen Datasets und der
einzelnen Observations folgen einem einheitlichen Muster, das von einer
\emph{Kennung} des Cubes ausgeht.  Für die Kennung \texttt{EH\_15G\_Plan14}
(Ergebnishaushalt Plandaten 2014 aus den Haushaltsdaten 2015/16) lauten die
URIs wie folgt:
\begin{itemize}\itemsep0pt
\item \texttt{http://haushaltsrechner.leipzig.de/Data/EH\_15G\_Plan14/} ist die
  URI des RDF Graphen, der auch als Turtle-Datei \texttt{EH\_15G\_Plan14.ttl}
  im git-Repo gespeichert ist;
\item \texttt{ihrds:EH\_15G\_Plan14} ist die URI des Datasets und 
\item \texttt{ihrdata:EH\_15G\_Plan14-Bezug1100111101} ist die URI einer
  Observation, die sich aus der gewählten Kennung und der Produktnummer -- in
  diesem Fall eines Schlüssel"|pro"|dukts -- zusammensetzt, auf die sich die
  jeweilige Observation bezieht.

  Da es in jedem Dataset zu jedem Produkt nur einen Datensatz gibt, wird auf
  diese Weise die eindeutige Referenzierbarkeit der Observations auch über der
  Grenzen des jeweiligen Datasets hinaus gewährleistet.
\end{itemize}
Unter der URI des RDF Graphen ist die genaue Primärdatenquelle auch noch einmal
als \texttt{rdfs:comment} angegeben. 

Eine \emph{Observation} hat typischerweise folgende Struktur:
\begin{code}
ihrdata:EH\_15G\_Plan14-Bezug1100111101\+\\
    ihr:relatesTo ihr:Bezug1100111101 ;\\
    ihr:jahr {\dq}2014{\dq};\\
    ihr:kategorie ihr:Ergebnishaushalt ;\\
    ihr:ein {\dq}2341177.24{\dq}; ihr:aus {\dq}-153900{\dq};\\
    ihr:waehrung dbpedia:Euro ;\\
    qb:dataSet ihrds:EH\_15G\_Plan14 ;\\
    a qb:Observation .
\end{code}
Die einzelnen Prädikate sind in der \emph{DatasetDefinition}
\texttt{ihr:DSDShort} definiert, die im DataSet \texttt{ihrds:EH\_15G\_Plan14}
referenziert wird:
\begin{itemize}\itemsep0pt
\item \texttt{ihr:relatesTo} -- eine \emph{DimensionProperty}, die auf die
  Produktnummer verweist, auf welche sich die \emph{Measure} bezieht,
\item \texttt{ihr:jahr} -- eine \emph{DimensionProperty}, die auf das Jahr
  verweist, auf welches sich die \emph{Measure} bezieht\footnote{Der Range
    dieser Property ist \texttt{xsd:gYear}, allerdings wird auf die
    Range-Angabe als Datentyp in den einzelnen Cubes verzichtet, da diese
    Angaben beim Ontowiki-Import aktuell falsch behandelt werden.},
\item \texttt{ihr:kategorie}  -- eine \emph{DimensionProperty}, die auf den
  Haushalttyp verweist, auf welchen sich die \emph{Measure} bezieht,
\item \texttt{ihr:ein}, \texttt{ihr:aus} -- zwei \emph{MeasureProperties},
  unter denen die Einnahmen und Ausgaben zu diesem Sachverhalt zahlenmäßig als
  \texttt{xsd:decimal} erfasst sind, 
\item \texttt{ihr:waehrung} -- eine \emph{AttributeProperty}, die angibt, auf
  welche Einheit sich die Maßzahlen beziehen.  Wert dieser Property ist in
  allen Fällen \texttt{dbpedia:Euro}, eine Referenz auf die Definition der
  Währungseinheit „Euro“ in der weit verbreiteten Dbpedia-Ontologie.  
\end{itemize}
In jedem Daten-Cube sind zusammen mit einer Observation zu einem Knoten im
Produktbaum auch zu jedem Vorgängerknoten eine Observation enthalten. Einnahmen
und Ausgaben in einer Observation zu einem Knoten im Inneren des Produktbaums
ergeben sich als Summe der Einnahmen bzw.\ Ausgaben aller Kindknoten zu diesem
Knoten.  \textbf{Die Einnahmen und Ausgaben sind also auf allen Ebenen des
  Produktbaums aus den Angaben in den Blättern kumuliert vorberechnet.}

\subsection{Das Produktmodell}

Wie bereits im Kapitel „Die Datenlage“ beschrieben ist unser Produktmodell der
Systematik des Ergebnishaushalts in einem \textbf{Produktgraphen} erfasst und
die genauere Charakterisierung der einzelnen Knoten dieses Graphen als
\textbf{Produktmodell} formalisiert.  Beide liegen als RDF-Graphen vor, der
Produktgraph in der Datei \texttt{NeuerProduktgraph.ttl} und das Produktmodell
in der Datei \texttt{NeuesProduktModell.ttl}.

Das Produktmodell umfasst die vier Ebenen Produktbereich, Produktgruppe,
Produktuntergruppe und Schlüsselprodukte.  Für jede dieser vier Ebenen ist ein
RDF-Typ definiert. Die URIs der Instanzen des jeweiligen Typs werden aus einem
typspezifischen Präfix und einer ID in der Form \texttt{ihr:<Präfix><ID>}
gebildet.  Die entsprechenden Informationen wurden für den Ergebnishaushalt
aus verschiedenen Quellen extrahiert und mit dem Produktplan
(S.~686\,ff. Haushaltsplan, Band~1) abgeglichen.

Die Details sind in der folgenden Tabelle zusammengestellt.
\begin{center}
  \begin{tabular}{rlll}
    Anzahl & Ebene & Präfix & RDF-Typ \\\hline
    32 & Produktbereiche & PrBer & \texttt{ihr:PrBer}\\
    92 & Produktgruppen & PrGr & \texttt{ihr:PrGr} \\
    127 & Produktuntergruppen & PrUGr & \texttt{ihr:PrUGr} \\
    282 & Schlüsselprodukte & Bezug & \texttt{ihr:Schluesselprodukt}\\
  \end{tabular}
\end{center}
Die Beziehungen zwischen den Elementen benachbarter Ebenen sind als Kanten des
(gerichteten) Produktgraphen durch das Prädikat \texttt{ihr:hasChild}
beschrieben.

Im Redesign wurden die Schlüsselprodukte konsequent von den PSP-Elementen
getrennt und für letztere ein weiterer RDF-Typ \texttt{ihr:PSP-Element}
eingeführt, der über ein Prädikat \texttt{ihr:belongsTo} einem Schlüsselprodukt
zugeordnet werden kann.  Damit besteht in Zukunft die Möglichkeit, die
Haushaltsdaten nicht nur auf der Ebene der Schlüsselprodukte zu kumulieren,
sondern auch noch den Bezug zur Ebene der PSP-Elemente genauer darzustellen.
Dies ist im aktuellen HH-Rechner allerdings noch nicht umgesetzt. 

Zu den drei Ebenen Produktbereich, Produktgruppe und Produktuntergruppe liegen
typischerweise Informationen in folgender Form vor:
\begin{code}
  ihr:PrUGr7520 a ihr:PrUGr ;\+\\
  rdfs:label {\dq}Schadensereignisse Bau- und Grundstücksordnung{\dq}@de ;\\
  ihr:BezeichnungStadt {\dq}Schadensereignisse Bau- und Grundstücksordnung{\dq} ;\\
  ihr:hatStadtId {\dq}7520{\dq} .
\end{code}

Hierbei sind den einzelnen Prädikaten folgende Informationen als Werte
zugeordnet:
\begin{itemize}
\item \texttt{ihr:PrUGr7520} die URI des Produkts, die sich in diesem Fall
  einer Produktuntergruppe aus dem Präfix \texttt{ihr:PrUGr} und der ID der
  Untergruppe in der Stadtsystematik zusammensetzt,
\item \texttt{ihr:PrUGr} der RDF-Typ des Produkts,
\item \texttt{rdfs:label} eine aus den Stadtdaten entnommene und
  ggf. korrigierte oder aus anderen Quellen konsolidierte Bezeichnung des
  Produkts,
\item \texttt{ihr:BezeichnungStadt} die aus den Stadtdaten übernommene genaue
  Bezeichnung mit allen Punkten, Abkürzungen und Sinnentstellungen, die sich
  aus den Größenbeschrän"|kungen der Felder in der Primärquelle ergeben. Diese
  Bezeichnung dient bei der Suche in Primärdaten als Fremdschlüssel, um Bezüge
  zwischen verschiedenen Datenquellen aufzudecken.
\item \texttt{ihr:hatStadtId} die Referenznummer des Produkts in den Stadtdaten.
\end{itemize}
Letzteres ist für Produktbereiche, Produktgruppen, Produktuntergruppen
identisch mit der aus der Stadtsystematik inferierten ID, für Schlüsselelemente
und PSP-Elemente wird die ID durch Entfernen der Punkte auf einen reinen
Zahlenwert reduziert -- in diesem Fall ist der Wert des Prädikats
\texttt{ihr:hatStadtId} die ursprüngliche Referenz mit Punkten.

Schlüsselelemente enthalten weitere Informationen wie in folgendem Beispiel:
\begin{code}
  ihr:Bezug1100111102 a ihr:Schluesselprodukt ;\+\\
  rdfs:label {\dq}Leitungshilfe und Unterstützung{\dq}@de ;\\
  ihr:BezeichnungStadt {\dq}Leitungshilfe und Unterstützung{\dq} ;\\
  ihr:hatStadtId {\dq}1.100.11.1.1.02{\dq} ;\\
  ihr:hatGrad {\dq}2{\dq} ;\\
  ihr:zumAmt ihr:Amt011 .
\end{code}

wobei 
\begin{itemize}
\item \texttt{ihr:hatGrad} den finanziellen Gestaltungsspielraum der Kommune
  bei diesem Schlüssel"|produkt angibt und 
\item \texttt{ihr:zumAmt} eine Referenz auf eine URI des Amts entsprechend der
  Ämterübersicht  (S. 634\,ff. Haushaltsplan, Band 1) ist, welchem dieses
  Schlüsselprodukt zugeordnet ist. 
\end{itemize} 
Der Gestaltungsspielraum $g$ ist eine Kommazahl im Bereich $1\le g\le 3$, wobei
\begin{center}
  $1$ für nicht beeinflussbar, $2$ für bedingt beeinflussbar und $3$ für stark
  beeinflussbar 
\end{center}
steht. Da $g$ in der Primärquelle
\texttt{Produktplan\_2015\_mit\_Spielraumangaben} teilweise nicht direkt den
Schlüsselprodukten zugeordnet war, sondern nur verschieden ausgezeichneten
Unterprodukten, sind die Werte $g$ für einige Schlüsselprodukte interpoliert
und dann auch nicht mehr ganzzahlig.  Da die Gestaltungsspielräume für
einzelne Knoten höherer Ebenen im Produktgraphen sowieso auf eine noch zu
bestimmende Weise aus den Gestaltungsspielräumen der Kindknoten zu berechnen
sind, ist diese Setzung naheliegend.

\subsection{Der Produktgraph}

Zur Aufstellung einer ersten Version des Produktgraphen wurden alle Zeilen mit
10-stelligen PSP-Nummern (Schlüsselprodukte) aus der Datei
\texttt{HH2014/EH\_14Ratsinfo.csv} extrahiert und die Verbindungen von
Produktbereich, Produktgruppe, Produktuntergruppe und Schlüssel"|pro"|dukt
extrahiert und mit dem Prädikat \texttt{ihr:hasChild} beschrieben.

Ein typischer Eintrag sieht wie folgt aus
\begin{code}
  ihr:PrBer42 ihr:hasChild ihr:PrGr421, ihr:PrGr424 .
\end{code}

und ist mit Verweis auf die obigen Ausführungen zum Produktmodell weitgehend
selbsterklärend. Dieser RDF Graph dient dazu, die Beziehungen zwischen den
verschiedenen Einheiten auf der Ebene der URIs darzustellen.  Er wurde im
weiteren Projektverlauf mehrfach mit anderen Quellen abgeglichen und auf diese
Weise konsolidiert. 

\section{Die Datentransformation der Primärdaten}

Dem Projekt wurden folgende Primärdaten in Form von Exceldateien zur Verfügung
gestellt, die aus dem SAP-System der Kämmerei extrahiert wurden:
\begin{itemize}\raggedright
\item Zu Projektbeginn im Oktober 2014 Plandaten zum Haushaltsansatz 2014,
  siehe Verzeichnis \texttt{Primaerdaten/HH2014},
\item im Mai 2015 Plandaten zum Haushaltsansatz 2015, siehe Verzeichnis
  \texttt{Primaerdaten/HH2015-1} und
\item Ende Juli 2015 Plandaten zu den Haushaltsansätzen 2015 und 2016, siehe
  Verzeichnis \texttt{Primaerdaten/HH2015-2}.
\end{itemize}
Diese Excel-Dateien wurden in der Regel nach folgendem Schema aufbereitet:
\begin{itemize}
\item Umwandlung der Excel-Datei in eine csv-Datei,
\item Extraktion relevanter Datensätze über das Kommando \texttt{grep}, das
  Zeilen nach vorgegebenem Muster (etwa „Planansatz2017“) aus der csv-Datei
  auswählt,
\item Transformation dieser Daten in RDF-Graphen mit verschiedenen
  Perl-Skripten\footnote{\url{https://de.wikipedia.org/wiki/Perl_(Programmiersprache)}}.
\end{itemize}
\subsection{Verzeichnis \texttt{Primaerdaten/HH2014}}
Dieses Verzeichnis enthält die zum Projektbeginn im Oktober 2014 übergebenen
Plandaten zum Haushaltsansatz 2014.
\begin{itemize}
\item \texttt{ErgHH\_PB11\_2014}

Detaillierte Planwertdaten zum Produktbereich 11 des Ergebnishaushalts für die
Jahre 2013 bis 2017. 

\item \texttt{FinHH} 

Datensätze zum Finanzhaushalt für die Jahre 2013 bis 2017.

\item \texttt{Ratsinfo} 

Datensätze mit Jahresvergleichen (1128 Datensätze zum Ergebnishaushalt, 3947
Datensätze zum Finanzhaushalt)
\end{itemize}
Die Datensätze aus \texttt{Ratsinfo} wurden zur weiteren Verarbeitung in
die Dateien 
\begin{quote}
  \texttt{EH\_14Ratsinfo.csv} (Datensätze zum Ergebnishaushalt) und\\
  \texttt{FH\_14Ratsinfo.csv} (Datensätze zum Finanzhaushalt)
\end{quote}
separiert. 

\subsection{Verzeichnis \texttt{Primaerdaten/HH2015-1}}
Dieses Verzeichnis enthält die im Mai 2015 übergebenen Plandaten zum
Haushaltsansatz 2015
\begin{itemize}
\item \texttt{ErgHH\_1} 

Detaillierte Planwertdaten zum Ergebnishaushalt für die Jahre 2014 bis 2018
sowie Rechnungsendbeträge 2013.
\item \texttt{FinHH\_lfd\_Verwaltungsteaetigkeit} 

Datensätze zum Finanzhaushalt aus laufender Verwaltungstätigkeit für die Jahre
2014 bis 2018 sowie Rechnungsendbeträge 2013.
\item \texttt{invest\_F8} 

Datensätze zum Investitionshaushalt für die Jahre 2014 bis 2018 sowie
Rechnungsergebnisse 2013.
\end{itemize}
Die Datensätze aus \texttt{ErgHH\_1} wurden zur weiteren Verarbeitung 
\begin{center}
  in die Dateien \texttt{EH\_15N\_Plan<j>.csv} mit $j\in \{14,15,16,17,18\}$
\end{center}
separiert, mit Blick auf die später als \texttt{HH2015-2} zur Verfügung
gestellten aktuelleren Daten gleicher Struktur aber nicht weiter verwendet.

\subsection{Verzeichnis \texttt{Primaerdaten/HH2015-2}}
Dieses Verzeichnis enthält die Ende Juli 2015 übergebenen Plandaten zum
Haushaltsansatz 2015 und 2016
\begin{itemize}
\item \texttt{ErgHH\_2015} und \texttt{ErgHH\_2016} 

Detaillierte Planwertdaten zum Ergebnishaushalt für die Jahre 2014 bis 2019
sowie Rechnungsendbeträge 2013 und 2014.
\item \texttt{FinHH\_lfd.Vw\_2015} und \texttt{FinHH\_lfd.Vw\_2016}

Datensätze zum Finanzhaushalt aus laufender Verwaltungstätigkeit für die Jahre
2014 bis 2019 sowie Rechnungsendbeträge 2013 und 2014.
\item \texttt{invest\_2015} und \texttt{invest\_2016} 

Datensätze zum Investitionshaushalt für die Jahre 2014 bis 2019 sowie
Rechnungsergebnisse 2013 und 2014.
\item \texttt{Ratsinfo\_2015} und \texttt{Ratsinfo\_2016} 

Datensätze mit Jahresvergleichen zum Ergebnishaushalt und zum Finanzhaushalt
\end{itemize}
Die Datensätze aus \texttt{ErgHH\_2015} und \texttt{ErgHH\_2016} wurden zur
weiteren Verarbeitung in die Dateien
\begin{center}
  \texttt{EH\_15G\_Plan<j>.csv} mit $j\in \{14,15,16,17,18,19\}$
\end{center}
separiert.  Die Plandaten zu einem Jahr, das in beiden Dateien erfasst ist,
waren identisch, so dass hier keine weitere Unterscheidung erforderlich war.

Die Datensätze zum Ergebnishaushalt aus \texttt{Ratsinfo\_2015} und
\texttt{Ratsinfo\_2016} wurden zur weiteren Verarbeitung
\begin{center}
  in die Dateien \texttt{EH\_15Ratsinfo.csv} und \texttt{EH\_16Ratsinfo.csv}
\end{center}
separiert.  Hierbei ist zu beachten, dass in den Ratsinfo-Dateien nur für die
Daten des jeweiligen Jahrs nach Einnahmen und Ausgaben unterschieden wird, die
Daten der Folgejahre dagegen nur in konsumierter Sicht vorliegen, wo Einnahmen
und Ausgaben zu einer Zahl zusammengezogen sind.  Deshalb können aus einer
solchen Datei auch nur Einnahmen und Ausgaben für das jeweils aktuelle Jahr
entnommen werden.

\subsection{Transformation}

Im Transformationsprozess für die Plandaten aus dem Ergebnishaushalt des
jeweiligen Jahres wurden die Zahlen aus den Spalten „Einnahmen“ und „Ausgaben“
sowie die PSP-Nummer übernommen. 
\begin{itemize}
\item Aus der PSP-Nummer wurden die zugehörigen URIs von „Schlüsselprodukt“,
  „Produktuntergruppe“, „Produktgruppe“ und „Produktbereich“ generiert.
\item Die Zahlen aus den Spalten „Einnahmen“ und „Ausgaben“ wurden in das
  internationale Zahlenformat transformiert.
\item Diese Zahlen wurden auf auf einem Hash mit den jeweiligen URIs als
  Schlüssel auf jeder Ebene des Produktbaums aggregiert.
\item Der Hash wurde ausgewertet, eine Observation pro Schlüssel erzeugt und
  als RDF Cube abgespeichert.
\end{itemize}
Der RDF Cube hat denselben Namen wie die Quelle, aus der er erzeugt wurde.  Für
den aktuellen Haushalt wird mit der Cube-Serie \texttt{EH\_15G\_Plan*}
gearbeitet, als zweite Serie steht \texttt{EH\_*Ratsinfo} zur Verfügung.

Für beide Serien weisen die Primärdaten aus der Kämmerei noch erhebliche
qualitative Defizite auf, so dass die Entscheidung über die Datenbasis, die im
HH-Rechner schließlich Verwendung findet, aktuell noch nicht möglich ist.  Die
Datenbasis des HH-Rechners kann in der Datei \texttt{Config.ttl} konfiguriert
werden.


\begin{thebibliography}{xxx}
\bibitem{RDFS} RDF Schema 1.1.  W3C Recommendation 25 February 2014.
  \url{http://www.w3.org/TR/rdf-schema/}.
\bibitem{RDF-Cube} The RDF Data Cube Vocabulary.  W3C Recommendation 16 January
  2014.  \url{http://www.w3.org/TR/vocab-data-cube/}.
\bibitem{SDMX} The Statistical Data and Metadata Exchange (SDMX) Initiative.
  \url{http://www.sdmx.org}.
\end{thebibliography}

\end{document}
