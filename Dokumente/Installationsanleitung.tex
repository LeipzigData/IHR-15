\documentclass[a4paper,11pt,twoside]{article}
% Encoding lokal, da manche Editoren danach schauen
\usepackage[utf8]{inputenc} 
\usepackage{ihr-15}

\newcommand{\Anmerkung}[1]{
  \begin{quote} \textbf{Anmerkung:}\ #1 \end{quote}
}

\title{Installationsanleitung „Interaktiver Haushaltsrechner“}
 
\date{Version vom 9. September 2015}

\begin{document}
\maketitle
\tableofcontents
\newpage
\seitezwei
\newpage

\section{Einleitung}
In diesem Dokument wird erklärt, welche serverseitigen Voraussetzungen für das
Aufsetzen des Prototyps erfüllt sein müssen, wie man eine Drupal-Instanz auf
diesem Server aufbaut, unsere Drupal-Anpassungen einspielt, welche weiteren
Module mit welchen Einstellungen wie verwendet werden sowie wie unser Code
funktioniert und genutzt wird.

Die Ausführungen gehen davon aus, dass der Leser mit den Konzepten von RDF und
RDF Data
Stores\footnote{\url{https://de.wikipedia.org/wiki/Resource_Description_Framework}}
sowie der SPAQRL
Anfragesprache\footnote{\url{https://de.wikipedia.org/wiki/SPARQL}}
hinreichend vertraut ist.

\section{Voraussetzungen}
In unserer Entwicklungsumgebung kamen
Drupal~7.36\footnote{\url{https://www.drupal.org/}} als CMS, MySQL~5 als
Datenbank und der Virtuoso Universal
Server\footnote{\url{http://www.openlinksw.com/}}~7 als RDF Store in einer
Linux OpenSuse Plattform zum Einsatz.  Drupal und Virtuoso sind für die
gängigen Betriebssysteme frei herunterladbar bzw.\ in Linux-Distributionen als
Pakete verfügbar.  

Der Prototyp sollte auch mit anderen gängigen Versionen von Drupal~7 und
Virtuoso laufen.  Weiterhin wird ein http-Server mit PHP-5.x Unterstützung
benötigt.

\subsection{Drupal}

Drupal~7 ist zwar ein webserverunabhänigiges Framework. Allerdings wird ein
Apache 1.3 oder 2.x empfohlen, da Drupal mit diesem Server entwickelt wurde
und somit mehr Erfahrungsberichte von Seiten der Community existieren, wodurch
man bei Problemen auf verschiedene Foren zurückgreifen kann. Für die
Installation sollte mindestens 60 MB Speicher auf dem Server frei sein und das
Modul \texttt{mod\_rewrite} muss innerhalb von Apache aktiviert sein. Da
Drupal in PHP geschrieben wurde, muss auf dem Server PHP (für Drupal~7.36
mindestens PHP~5.2) vorhanden sein. Als Datenbankanbindungen kommen MySQL~5
sowie PostgreSQL~7.4 in Frage. Wir arbeiteten mit MySQL.

\subsection{Virtuoso}

Virtuoso ist eine RDF Data Store Engine mit einem eigenen SPARQL-Endpunkt.
Bei der Installation eines neuen Virtuoso Images wird ein erster Benutzer
(Standard: 'dba', mit Passwort 'dba') angelegt, dessen Passwort unbedingt
angepasst werden sollte.

Zugriff und Verwaltung der Daten ist über die Konsole und einen
\emph{Datenport} (Standard 1111) oder über einen \emph{Webport} auf dem
Web-Server (Standard 8890) möglich, der über den \emph{Conductor}
administriert werden kann.  Letzterer ist unter
\texttt{\textit{[Serverroot]}:\textit{[webport]}} erreichbar und nach
Authentifizierung mit den Datenbank-Zugangsdaten voll nutzbar.  Der
SPARQL-Endpunkt ist über die Webschnittstelle
\texttt{\textit{[Serverroot]}:\textit{[webport]}/sparql} zu erreichen. 

Im Projekt wurden beide Zugangsformen zur Datenverwaltung verwendet. 

\section{Daten}

Die Datentripel liegen als RDF Data Cube Serien im Verzeichnis \texttt{Daten}
der Distribution und müssen in den Virtuoso Triplestore hochgeladen werden.
Details zur Struktur dieser Daten sind im Dokument \emph{Designprinzipien des
  IHR-15 RDF Data Stores} zu finden.

\Anmerkung{Hierzu ist noch ein Upload-Skript zu erstellen, mit dem die Serie
  gleich als Ganzes über die Konsole hochgeladen werden kann.}

Nachdem der Virtuoso Universal Server aufgesetzt ist, können die Daten über
den Conductor eingespielt werden. Über den Menüpunkt \textit{Linked Data $>$
  Quad Store Upload} kommt man zur Uploadansicht.  Für jeden einzelnen RDF
Graphen (dieser ist identisch mit einer *.ttl Datei) muss dazu die
entsprechende Datei ausgewählt \emph{und} der \textit{Graph Name} auf den in
der Datei definierten Wert gesetzt werden. 

Beide Größen sind durch ein einfaches Namensschema miteinander verbunden.  Die
Turtle Datei \texttt{Daten/EH\_15G\_Plan15.ttl} etwa enthält den Graphen 
\begin{quote}\tt
  http://haushaltsrechner.leipzig.de/Data/EH\_15G\_Plan15/
\end{quote}

Die Namen der Graphen der verwendeten RDF Data Cube Serie sind in der Datei
\texttt{Config.ttl} aufgelistet, die ebenfalls als RDF Graph 
\begin{quote}\tt
  http://haushaltsrechner.leipzig.de/Data/Config/
\end{quote}
in den RDF Store zu laden ist.  Diese Information wird vom Software-Controller
des IHR-Prototyps ausgelesen.

\section{Drupal-Installation -- vereinfacht}
Um die Installation zu vereinfachen, haben wir ein Paket zusammengestellt, in
dem bereits alle von uns verwendeten Module, Themes und Sprachanpassungen
enthalten sind. Außerdem ist in dem Paket ein Datenbankbackup enthalten, das
über das Drupal-Modul \emph{Backup and Migrate} eingespielt werden kann.
Dadurch wird die Drupal-Instanz innerhalb weniger Schritte mit unseren
Anpassungen und Inhalten installiert. Das konkrete Vorgehen wird im Folgenden
beschrieben.

\subsection{Step-By-Step-Anleitung}
\begin{enumerate}
\item Entpacken Sie das Paket, benennen Sie den entpackten Ordner ggf.\ nach
  ihren Vorstellungen um und laden Sie ihn an den gewünschten Ort
  \textit{[Drupalfolder]} auf dem Server hoch.
\item Setzen Sie die Rechte des Ordners
  \texttt{\textit{[Drupalfolder]}/sites/default} auf 777 und die Rechte der
  Datei \texttt{\textit{[Drupalfolder]}/sites/default/settings.php} auf 666
  (chmod).
\item Rufen Sie die Adresse des Drupalordners
  (z.\,B.\ \textit{http://www.xyz.org/Drupal/}) im Browser auf, um die
  Installation zu starten.
\item Wählen Sie die Standard-Installation, dann als Sprache \emph{German} und
  geben Sie schließlich die Daten Ihrer MySQL-Datenbank und die zuvor
  angelegte Admin-Mailadresse ein. Bestätigen Sie diese Daten und warten Sie,
  bis die Installation abgeschlossen ist.
\item Setzen Sie die Rechte des Ordners
  \texttt{\textit{[Drupalfolder]}/sites/default} wieder auf 755 und die Rechte
  der Datei \texttt{\textit{[Drupalfolder]}/sites/default/settings.php} wieder
  auf 644 (chmod).
\item Rufen sie nun erneut die Adresse des Drupal-Ordners auf. Hier sollte
  Ihnen nun ihre neu installierte Drupal-Instanz angezeigt werden.
\item Loggen Sie sich als Administrator ein, rufen Sie die Modul-Seite auf und
  aktivieren Sie das Modul \emph{Backup and Migrate}.
\item Navigieren Sie über das Menü zu \emph{Konfiguration $>$ System $>$
  Backup and Migrate $>$ Einstellungen $>$ Destinations $>$ Add Destination}.
  Klicken Sie hier auf \emph{Server Directory} und geben dann einen beliebigen
  Namen und den Ordner \texttt{./sites/default/} an.
\item Wählen Sie nun den Reiter \emph{Restore} und auf der folgenden Seite
  \emph{Restore from a saved backup}. Klicken Sie dann auf \emph{Restore now}.
\item Rufen Sie nun noch die Seite \texttt{\textit{[Drupalfolder]}/clear.php}
  auf.  Löschen Sie diese Datei im Anschluss.
\end{enumerate}

\section{Drupal manuell installieren und konfigurieren}
\subsection{Drupal aufsetzen}

Drupal 7.36 diente in diesem Projekt als Basis-Framework. Zum Herunterladen
und Installieren von Drupal kann die Dokumentation unter
\url{https://www.drupal.org/} zu Hilfe gezogen werden. Da die Dokumentation
dort zur grundlegenden Installation ausreichend ist, soll hier nicht weiter
darauf eingegangen werden.

\subsection{Module}
Um den Funktionsumfang von Drupal auf unsere Bedürfnisse anzupassen, müssen
einige zusätzliche Module installiert werden. Diese können über die Webseite
\url{https://www.drupal.org/} heruntergeladen und eingefügt werden. Eine
Anleitung hierzu findet sich unter
\url{https://www.drupal.org/documentation/install/modules-themes}.  Die Module
sind alle unter der GNU GPL erhältlich.  

Folgende Module sind zu installieren: 
\begin{enumerate}
\item \emph{jQuery Update}

Dieses Modul stellt die aktuelle Version von jQuery zur Verfügung, die für
\textit{bootstrap} benötigt wird.
\item \emph{advanced forum}

Das Advanced Forum ergänzt das von Haus aus in Drupal integrierte Forum um
einige Funktionen, die bei anderen Forensoftwares standardmäßig enthalten
sind. Nach Aktivierung des Moduls können über \emph{Struktur $>$ Foren} nach
der gewünschten Gliederung Foren und Ordner, in die Foren eingeordnet werden
können, erstellt werden. Unter \emph{Konfiguration $>$ Inhaltserstellung $>$
  Advanced Forum} können grundlegende Einstellungen angepasst werden, wie das
Design des Forums. Im Projekt wurde das \emph{Bootstrap}-Theme verwendet,
welches wiederum ein eigenes Modul darstellt.
\item \emph{advanced forum bootstrap}

Das für unser Projekt verwendete Design für das Forum -- setzt \emph{Advanced
  Forum} voraus.

\item \emph{flag / flag abuse}

Diese Pakete dienen zum Erstellen von Flags für das Melden von Kommentaren und
Beiträgen. Dieses Paket ist notwendig zur Erstellung von Flags für das Melden
von Beiträgen im Forum. Hierzu wurden im Projekt Flags \textit{Comment Abuse}
aus den Standardflags und die die eigens erstellte Flag \textit{Topic Abuse}
zum Melden von Kommentaren bzw.\ Themen im Forum verwendet. Zum Erstellen einer
Flag dient der Menü-Eintrag \textit{Struktur $>$ Flags $>$ Add Flags}.
\item \emph{views}

Mit Views lassen sich Ansichten gefilterter Inhalte erstellen. Zum Beipiel zum
Anzeigen aller gemeldeten Kommentare oder Themen. Das Erstellen einer View sei
hier am Beispiel des Views für gemeldete Kommentare erklärt.

In der Adminansicht kann unter dem Menüpunkt \textit{Struktur $>$ Ansichten
  $>$ Neue Ansicht hinzufügen} eine neue Ansicht erstellt werden. Zu dieser
neuen Ansicht kann im geöffneten Dialog ein \textit{Ansichtsname} und ein Pfad
angegeben werden, unter dem die Ansicht später erreichbar ist.  In unserem
Fall wurde der Pfad \textit{gemeldete-kommentare} gewählt. Nach Klick auf
\textit{Fortfahren und Bearbeiten} wird festgelegt, welcher Inhalt auf diesem
\textit{View} angezeigt wird. Für die Ansicht aller gemeldeten Kommentare
bietet sich das Format \textit{Tabelle} an. Dann kann unter \textit{Felder}
festgelegt werden, welche Eigenschaften der Kommentare in der View angezeigt
werden. Im Projekt wurden \textit{Kommentar: Titel}, \textit{Flag:
  Kennzeichnungs-Link} und \textit{Flags: Kennzeichnungs-Zeitpunkt}
gewählt. Weiterhin muss unter dem Menüpunkt \textit{Erweitert $>$ Beziehungen}
eine Beziehung zur Melde-Flag hinzugefügt werden. Mit Klick auf
\textit{Hinzufügen} kann der Eintrag \textit{Flags:Kommentar Flag} ausgewählt
werden. Im folgenden Fenster kann man noch einen Bezeichner für die Flag
einführen, die Kennzeichnung Abuse und den Punkt \textit{beliebiger Benutzer}
anwählen. Anschließend kann unter Filterkriterien ausgewählt werden, dass nur
solche Kommentare angezeigt werden, die geflagt wurden, indem man den Eintrag
\textit{Flag Kennzeichnung} hinzufügt.
\item \emph{printfriendly}
\item \emph{votingapi}

Hier können wichtige Einstellungen für das \emph{Rate Widget} bestimmt werden.
Es kann bestimmt werden, wann anonyme Nutzer erneut von einem Rechner wählen
könnne (wenn sie es denn dürfen -- dies wird für jedes Rate Widget einzeln
bestimmt).  Gleiches kann für registrierte Nutzer bestimmt werden. Bei hoher
Serverlast können auch die Berechnungszeiten der Bewertungen auf bestimmte
Cron-Zeiten festgelegt werden.
\item \emph{rate}

Mit \emph{rate} können verschiedene Beiträge bewertet werden. Um ein neues
\emph{Rate Widget} zu erstellen, geht man auf \textit{Struktur $>$ Rate
  Widgets}.  Dort kann man neue \emph{Rate Widgets} bearbeiten und erstellen.
Es kann Name, Tag, Inhaltstyp etc.\ festgelegt werden.

\item \emph{statistics counter}

Zusätzlich zur Installation sollte unter \textit{Konfiguration $>$ System $>$
  Statistiken} die Optionen \textit{Zugriffsprotokoll aktivieren} und
\textit{Inhaltsabrufe zählen} aktiviert werden. Danach können
Besucherstatistiken unter dem Menüpunkt \textit{Berichte} eingesehen werden.

\item \emph{relation, ctools}

Helper-Module, die für Funktionen der anderen, essentiellen Module benötigt
werden.
\end{enumerate}

\subsection{IHR-Theme hinzufügen}
Das Design der Website basiert auf dem freien CSS Framework \emph{Bootstrap},
das Gestaltungsvorlagen für Typografie, Formulare, Buttons, Tabellen etc.\ und
optionale JavaScript-Erweite"|rungen enthält. Für die Projektseite wurde das
Basis-Theme \url{https://www.drupal.org/project/bootstrap} verwendet, das die
grundlegenden Elemente des Frameworks in die Funktionalität von Drupal
einbindet.  Zur besseren Annäherung an die offizielle Website der Stadt wurde
ein \emph{Subtheme} erstellt, welches das oben genannte Theme ergänzt.

Zur Aktivierung des Themes reicht es, das Bootstrap-Theme gemeinsam mit dem
IHR-Theme in den \texttt{theme}-Ordner der installierten Drupal-Instanz zu
laden (Standard: \texttt{/themes/}) und anschließend im
Design-Administrationsmenü zu aktivieren.

\subsubsection{Menüs erstellen}
Die Menüs der Website werden über den Punkt \textit{Struktur $>$ Menüs}
administriert. Dabei stellt das Menü mit dem Namen "`Main menu"' das Hauptmenü
im Kopfbereich der Seite dar -- zusätzlich dazu kann ein Menü in der Fußleiste
erstellt werden, in dem Links wie das Impressum, Datenschutzerklärung und
Kontaktformular abgelegt werden können.

Dazu muss unter oben genanntem Menü auf \emph{Menü hinzufügen} geklickt werden
und der Titel so gewählt werden, dass im rechten Bereich der maschinenlesbare
Name \textit{menu-footer-menu} entsteht. Danach können Links hinzugefügt
werden, die dann im rechten unteren Fußbereich der Website erscheinen.

\subsubsection{Pages erstellen}
Im Releasepaket sind im Ordner \emph{IHR-Pages} drei Dateien enthalten, die
nach Installation von Drupal über \textit{Inhalt $>$ Inhalt hinzufügen $>$
  Basic Page} eingepflegt werden können. In \textit{EinnahmenDia.html} und
\textit{AusgabenDia.html} ist der Code für die beiden Diagramme enthalten.
Wichtig ist hierbei, als Textformat \emph{Full HTML} auszuwählen.

In der Datei \texttt{indexpage.php} ist der Inhalt der bei Projektende
erstellten Startseite enthalten. Hier ist bei Hinzufügen \emph{PHP Code}
auszuwählen.

\end{document}
