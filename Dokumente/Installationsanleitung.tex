\documentclass[a4paper,11pt,twoside]{article}
% Encoding lokal, da manche Editoren danach schauen
\usepackage[utf8]{inputenc} 
\usepackage{ihr-15}

\newcommand{\Anmerkung}[1]{
  \begin{quote} \textbf{Anmerkung:}\ #1 \end{quote}
}

\title{Installationsanleitung „Interaktiver Haushaltsrechner“}
 
\date{Version vom 20. April 2016}

\begin{document}
\maketitle
\tableofcontents
\newpage
\seitezwei
\newpage

\section{Einleitung}
In diesem Dokument wird erklärt, 
\begin{itemize}\itemsep0pt
\item welche serverseitigen Voraussetzungen für das Aufsetzen des Prototyps
  erforderlich sind, 
\item wie man eine Drupal-Instanz auf diesem Server aufbaut, 
\item wie unsere Drupal-Anpassungen einspielt werden, 
\item welche weiteren Module mit welchen Einstellungen wie verwendet werden und
\item welche weiteren Konfigurationsschritte erforderlich sind.
\end{itemize}

\section{Voraussetzungen}
In unserer Entwicklungsumgebung kamen
Drupal~7.36\footnote{\url{https://www.drupal.org/}} als CMS, MySQL~5 als
Datenbank für Drupal und der Virtuoso Universal
Server\footnote{\url{http://www.openlinksw.com/}}~7.0.0 als RDF Store in einer
Linux OpenSuse Plattform zum Einsatz.  Drupal und Virtuoso sind für die
gängigen Betriebssysteme frei herunterladbar bzw.\ in Linux-Distributionen als
Pakete verfügbar.

Der Prototyp sollte auch mit anderen gängigen Versionen von Drupal~7 und
Virtuoso~6 oder 7 laufen.  Weiterhin wird ein für diese Anwendungen
konfigurierter http-Server mit PHP-5.x Unterstützung benötigt.

Der Prototyp verwendet eine Reihe von Drupal-Modulen in unveränderter Form.
Diese Module können unkompliziert durch aktualisierte Versionen ersetzt
werden.  Alle Drupal-Anpassungen sind im Subtheme \emph{IHR} enthalten, das
zusammen mit dem Basistheme \emph{bootstrap} im Theme-Bereich ausgerollt und
aktiviert werden muss. Danach ist die Plattform entsprechend zu konfigureren
(Menüs und Seiten). 

\subsection{Drupal}

Drupal~7 ist zwar ein webserverunabhänigiges Framework, allerdings wird Apache
1.3 oder 2.x als Webserver empfohlen, da Drupal für diese Server optimiert
wurde und auch mehr Erfahrungsberichte von Seiten der Community existieren,
wodurch man bei Problemen auf verschiedene Foren zurückgreifen kann. Für die
Installation sollte mindestens 60 MB Speicher auf dem Server frei sein. Das
Modul \texttt{mod\_rewrite} muss innerhalb von Apache aktiviert sein. Da Drupal
in PHP geschrieben ist, muss auf dem Server PHP (für Drupal~7.36 mindestens
PHP~5.2) mit den üblichen Erweiterungen für LAMP-Anwendungen\footnote{LAMP ist
  ein Kurzwort für eine standardisierte Serverumgebungen mit Linux, Apache,
  MySQL, PHP.  Meist ist klar, wie dafür entwickelte Applikationen an andere
  gängige Serverumgebungen angepasst werden können. Siehe dazu auch
  \url{https://de.wikipedia.org/wiki/LAMP_(Softwarepaket)}.}  installiert
sein. Als Datenbankanbindungen kommen MySQL~5 sowie PostgreSQL~7.4 in
Frage. Wir arbeiteten mit MySQL.

\subsection{Virtuoso}

Virtuoso ist eine RDF Data Store Engine mit einem eigenen SPARQL-Endpunkt.
Bei der Installation eines Virtuoso Images für einen neuen Data Store wird ein
erster Benutzer (Standard: 'dba', mit Passwort 'dba') angelegt, dessen
Passwort unbedingt anzupassen ist.  Passwörter werden von Virtuoso auf der
Ebene von Data Stores vergeben, es können also verschiedene Data Stores unter
demselben Benutzer 'dba' angelegt und eigenständig verwaltet werden, was
üblicherweise auch so praktiziert wird. Weitere Hinweise zur Installation und
Konfiguration von Virtuoso sowie (optional) eines OntoWiki Web-Frontends auf
einem lokalen Host sind unter
\url{http://wiki.symbolicdata.org/LocalSparqlEndpoint} zu finden.

Zugriff und Verwaltung der Daten ist über die Konsole und einen
\emph{Datenport} (Standard 1111) oder über einen \emph{Webport} auf dem
Web-Server (Standard 8890) möglich, der über den \emph{Conductor} administriert
werden kann.  Letzterer ist unter
\texttt{\textit{[Serverroot]}:\textit{[webport]}} erreichbar und nach
Authentifizierung mit den Zugangsdaten des Data Stores voll nutzbar.  Der
SPARQL-Endpunkt ist über die Webschnittstelle
\texttt{\textit{[Serverroot]}:\textit{[webport]}/sparql} zu erreichen.

\section{Daten}

Die Datentripel liegen als RDF Data Cube Serien im Verzeichnis \texttt{Daten}
der Distribution und müssen in den Virtuoso Triplestore hochgeladen werden.
Details zur Struktur dieser Daten sowie die genaue Beschreibung des mit diesem
Release ausgelieferten Datenbündels sind im Dokument \emph{Designprinzipien des
  IHR-15 RDF Data Stores} zu finden.

Die Daten in unserem Release zeigen nur beispielhaft, wie die Applikation
funktioniert, da es bis zum Projektende nicht gelungen ist, vom Dezernat für
Finanzen der Stadt Leipzig widerspruchsfreie und konsolidierte Daten zum
Haushalt der Stadt Leipzig zu bekommen. Der Datenimport in die Applikation ist
deshalb auch nur provisorisch über ein einfaches Perl-Skript
\texttt{Daten/Werkbank/loaddata.pl} realisiert, mit dem eine Eingabedatei
erzeugt und an die Virtuoso-Konsole \texttt{iswql-vt} (Ubuntu Linux, in anderen
Serverumgebungen heißt diese Virtuoso-Konsole ggf.\ anders) weitergereicht
werden kann.  Die Parameter in diesem Skript sind an die konkrete
Verzeichnisstruktur der Zielplattform entsprechend anzupassen.

Alternativ können die Daten über den Virtuoso-Conductor eingespielt werden.
Über den Menüpunkt \textit{Linked Data $>$ Quad Store Upload} kommt man zur
Uploadansicht.  Für jeden einzelnen RDF Graphen (dieser ist identisch mit einer
*.ttl Datei) muss dazu die entsprechende Datei ausgewählt \emph{und} der
\textit{Graph Name} auf den in der Datei definierten Wert gesetzt werden.
Details erschließen sich aus der Ausgabe von \texttt{loaddata.pl}.

Der SPARQL Endpunkt des RDF Stores auf dem Praktikumsserver ist als privater
Parameter \texttt{private \$endpoint} des Konstruktors
\texttt{EndpointHandler()} in der Datei
\begin{center}
  \texttt{code/IHR/data/modelcontroller.php}
\end{center}
hart codiert.  Für einen anderen RDF Store muss das entsprechend angepasst
werden.
\newpage

\section{Drupal-Installation -- vereinfacht}
Um die Installation zu vereinfachen, ist im Release die zip-Datei
\texttt{drupalDump.zip} enthalten, in der bereits alle von uns verwendeten
Module, Themes und Sprachanpassungen zusammengepackt sind. Außerdem ist in dort
ein Datenbankbackup enthalten, das über das Drupal-Modul \emph{Backup and
  Migrate} eingespielt werden kann.  Dadurch wird die Drupal-Instanz innerhalb
weniger Schritte mit unseren Anpassungen und Inhalten installiert. Das konkrete
Vorgehen wird im Folgenden beschrieben.

\subsection{Step-By-Step-Anleitung}
\begin{enumerate}
\item Entpacken Sie die zip-Datei und verschieben Sie den Ordner an den
  gewünschten Ort \textit{[Drupalfolder]} auf dem Server.
\item Setzen Sie die Rechte des Ordners
  \texttt{\textit{[Drupalfolder]}/sites/default} auf 777 und die Rechte der
  Datei \texttt{\textit{[Drupalfolder]}/sites/default/settings.php} auf 666
  (chmod).
\item Rufen Sie die Adresse des Drupalordners
  (z.\,B.\ \textit{http://www.xyz.org/Drupal/}) im Browser auf, um die
  Installation zu starten.
\item Wählen Sie die Standard-Installation, dann als Sprache \emph{German} und
  geben Sie schließlich die Daten Ihrer MySQL-Datenbank und die zuvor
  angelegte Admin-Mailadresse ein. Bestätigen Sie diese Daten und warten Sie,
  bis die Installation abgeschlossen ist.
\item Setzen Sie die Rechte des Ordners
  \texttt{\textit{[Drupalfolder]}/sites/default} wieder auf 755 und die Rechte
  der Datei \texttt{\textit{[Drupalfolder]}/sites/default/settings.php} wieder
  auf 644 (chmod).
\item Rufen sie nun erneut die Adresse des Drupal-Ordners auf. Hier sollte
  ihre neu installierte Drupal-Instanz angezeigt werden.
\item Loggen Sie sich als Administrator ein, rufen Sie die Modul-Seite auf und
  aktivieren Sie das Modul \emph{Backup and Migrate}.
\item Navigieren Sie über das Menü zu \emph{Konfiguration $>$ System $>$
  Backup and Migrate $>$ Einstellungen $>$ Destinations $>$ Add Destination}.
  Klicken Sie hier auf \emph{Server Directory} und geben dann einen beliebigen
  Namen und den Ordner \texttt{./sites/default/} an.
\item Wählen Sie nun den Reiter \emph{Restore} und auf der folgenden Seite
  \emph{Restore from a saved backup}. Klicken Sie dann auf \emph{Restore now}.
\item Rufen Sie nun noch die Seite \texttt{\textit{[Drupalfolder]}/clear.php}
  auf.  Löschen Sie diese Datei im Anschluss.
\end{enumerate}
Damit wird ein kompletter Dump unserer Entwicklungsinstanz auf dem Stand des
finalen Release eingespielt. Zu beachten ist, dass dabei auch alle Accounts,
Forenbeiträge etc.\ übernommen werden. Diese müssen in einem letzten Schritt
den lokalen Gegebenheiten angepasst werden. \textbf{Insbesondere sollten alle
  zusätzlichen Accounts mit Administratorrechten angepasst oder gelöscht
  werden.} Auch muss nach dem Einspielen des Backups ggf.\ die
System-Mailadresse geändert werden, da auch diese Information aus dem Backup
übernommen wird.

\section{Drupal manuell installieren und konfigurieren}

\subsection{Drupal aufsetzen}

In unserem Projekt wurde Drupal 7.36 als Basis verwendet, aber andere Versionen
von Drupal~7 können ebenfalls eingesetzt werden.  Zum Herunterladen und
Installieren von Drupal verweisen wir auf die ausführliche Dokumentation unter
\url{https://www.drupal.org/} und gehen im Weiteren von einer solchen
Basisinstallation von Drupal aus.

Eine deutsche Menüführung wird installiert, indem die Sprachdatei
\texttt{drupal-xx.de.po}\footnote{Eine aktuelle Version ist im
  Repo-Verzeichnis \texttt{Drupal-Modules} zu finden.} im Ordner
\texttt{profiles/standard/translations} einfügt wird.

\subsection{Module}
Um den Funktionsumfang von Drupal auf unsere Bedürfnisse anzupassen, müssen
einige zusätzliche Module installiert werden. Diese können ebenfalls über die
Webseite 
\begin{center}
  \url{https://www.drupal.org/}
\end{center}
heruntergeladen und eingefügt werden\footnote{Aktuelle Versionen dieser Module
  sind auch im Repo-Verzeichnis \texttt{Drupal-Module} zu finden.}.  Zum
Installieren eines Moduls muss die jeweilige zip-Datei in das Verzeichnis
\texttt{sites/all/modules} kopiert und dort entpackt werden.  Verzeichnisse
von Vorläuferversionen sollten vorher gelöscht werden.  Eine detailliertere
Anleitung hierzu findet sich unter
\begin{center}
  \url{https://www.drupal.org/documentation/install/modules-themes}.
\end{center}
Die Module stehen lizenzrechtlich alle unter der GNU GPL.

Folgende Module sind zu installieren: 
\begin{enumerate}

\item \emph{jQuery Update}

Dieses Modul stellt die aktuelle Version von jQuery zur Verfügung, die für
\textit{bootstrap} benötigt wird.

\item \emph{advanced forum}

Das Advanced Forum ergänzt das von Haus aus in Drupal integrierte Forum um
einige Funktionen, die bei anderen Forensoftwares standardmäßig enthalten sind.

Nach Aktivierung des Moduls können über \emph{Struktur $>$ Foren} nach der
gewünschten Gliederung Foren und Ordner, in die Foren eingeordnet werden
können, erstellt werden. Unter \emph{Konfiguration $>$ Inhaltserstellung $>$
  Advanced Forum} können grundlegende Einstellungen angepasst werden, wie das
Design des Forums. Im Projekt wurde das \emph{Bootstrap}-Theme verwendet,
welches wiederum ein eigenes Modul darstellt.

\item \emph{entity}\footnote{2016-04-20, Gräbe: Ergänzt, da in der
  Installation verwendet.}

This module extends the entity API of Drupal core in order to provide a
unified way to deal with entities and their properties. Additionally, it
provides an entity CRUD controller, which helps simplifying the creation of
new entity types.

\item \emph{flag / flag abuse}

Diese Pakete dienen zum Erstellen von Flags für das Melden von Kommentaren und
Beiträgen und kommt für das Melden von Beiträgen im Forum zum Einsatz.

Dazu werden Flags \textit{Comment Abuse} aus den Standardflags und die eigens
erstellte Flag \textit{Topic Abuse} zum Melden von Kommentaren bzw.\ Themen im
Forum verwendet. Zum Erstellen einer Flag dient der Menü-Eintrag
\textit{Struktur $>$ Flags $>$ Add Flags}.

\item \emph{views}

Mit Views lassen sich Ansichten gefilterter Inhalte erstellen, zum Beipiel zum
Anzeigen aller gemeldeten Kommentare oder Themen. Das Erstellen einer View sei
hier am Beispiel des Views für gemeldete Kommentare erklärt.

In der Adminansicht kann unter dem Menüpunkt \textit{Struktur $>$ Ansichten
  $>$ Neue Ansicht hinzufügen} eine neue Ansicht erstellt werden. Zu dieser
neuen Ansicht kann im geöffneten Dialog ein \textit{Ansichtsname} und ein Pfad
angegeben werden, unter dem die Ansicht später erreichbar ist.  In unserem
Fall wurde der Pfad \textit{gemeldete-kommentare} gewählt. Nach Klick auf
\textit{Fortfahren und Bearbeiten} wird festgelegt, welcher Inhalt auf diesem
\textit{View} angezeigt wird. Für die Ansicht aller gemeldeten Kommentare
bietet sich das Format \textit{Tabelle} an. Dann kann unter \textit{Felder}
festgelegt werden, welche Eigenschaften der Kommentare in der View angezeigt
werden. Im Projekt wurden \textit{Kommentar: Titel}, \textit{Flag:
  Kennzeichnungs-Link} und \textit{Flags: Kennzeichnungs-Zeitpunkt}
gewählt. Weiterhin muss unter dem Menüpunkt \textit{Erweitert $>$ Beziehungen}
eine Beziehung zur Melde-Flag hinzugefügt werden. Mit Klick auf
\textit{Hinzufügen} kann der Eintrag \textit{Flags:Kommentar Flag} ausgewählt
werden. Im folgenden Fenster kann man noch einen Bezeichner für die Flag
einführen, die Kennzeichnung Abuse und den Punkt \textit{beliebiger Benutzer}
anwählen. Anschließend kann unter Filterkriterien ausgewählt werden, dass nur
solche Kommentare angezeigt werden, die geflagt wurden, indem man den Eintrag
\textit{Flag Kennzeichnung} hinzufügt.

\item \emph{pathauto}

Damit werden automatisch URL/Pfad-Aliase für verschiedene Inhaltsarten
erstellt.

\item \emph{printfriendly}

Funktionalität hinter dem Button „Als PDF ausdrucken“, mit der Drupal Webseiten
ohne speziellen Code, css oder \texttt{print.css} auch für andere
Druckansichten aufbereitet werden.

\item \emph{votingapi}

Standardisierte Drupal-API für Bewertungen, die ein einheitliches Daten- und
Prozessmodell für Bewertungsprozesse von Drupal-Inhalten anbietet.

Hier können wichtige Einstellungen für das \emph{Rate Widget} vorgenommen
werden.  Es kann bestimmt werden, wann anonyme Nutzer erneut von einem Rechner
wählen könnne (wenn sie es denn dürfen -- dies wird für jedes Rate Widget
einzeln bestimmt).  Gleiches kann für registrierte Nutzer bestimmt werden. Bei
hoher Serverlast können auch die Berechnungszeiten der Bewertungen auf
bestimmte Cron-Zeiten festgelegt werden.

\item \emph{rate}

Mit \emph{rate} können verschiedene Beiträge bewertet werden. 

Um ein neues \emph{Rate Widget} zu erstellen, geht man auf \textit{Struktur $>$
  Rate Widgets}.  Dort kann man neue \emph{Rate Widgets} bearbeiten und
erstellen.  Es kann Name, Tag, Inhaltstyp etc.\ festgelegt werden.

\item \emph{statistics counter}

Kleine Erweiterung der grundlegenden Statistik-Funktionen im Drupal Core. 

Zusätzlich zur Installation muss unter \textit{Konfiguration $>$ System $>$
  Statistiken} die Optionen \textit{Zugriffsprotokoll aktivieren} und
\textit{Inhaltsabrufe zählen} aktiviert werden. Danach können
Besucherstatistiken unter dem Admin-Menüpunkt \textit{Berichte} eingesehen
werden.

\item \emph{relation, ctools}

Helper-Module, die für Funktionen der anderen, essentiellen Module benötigt
werden.

\end{enumerate}


\subsection{Themes hinzufügen}
Das Design der Website basiert auf dem freien CSS Framework \emph{Bootstrap},
das Gestaltungsvorlagen für Typografie, Formulare, Buttons, Tabellen etc.\ und
optionale JavaScript-Erweite"|rungen enthält. Für die Projektseite wurde das
Basis-Theme \url{https://www.drupal.org/project/bootstrap}
verwendet\footnote{Eine aktuelle Version ist im Repo-Verzeichnis
  \texttt{Drupal-Modules} zu finden.}, das die grundlegenden Elemente des
Frameworks in die Funktionalität von Drupal einbindet, und durch das Subtheme
\texttt{IHR} erweitert. Dieses enthält
\begin{itemize}\itemsep0pt
\item die grundlegenden Erweiterungen des Prototypen wie etwa die
  Diagrammdarstellungen und die Anbindung des RDF Data Stores sowie 
\item die Modifikationen von Bootstrap zur Anpassung an das Look and Feel der
  offiziellen Website der Stadt.
\end{itemize}
Als weiteres Theme wird \texttt{Forum-Theme} verwendet, das Erweiterungen und
Modifikationen des Moduls „Advanced Forum“ enthält. 

Zur Aktivierung eines Themes ist das Basis-Theme gemeinsam mit dem Subtheme in
den \texttt{theme}-Ordner der installierten Drupal-Instanz zu laden (Standard:
\texttt{/themes/}) und anschließend im Design-Administrationsmenü zu
aktivieren.

\subsubsection{Menüs erstellen}
Die Menüs der Website werden über den Punkt \textit{Struktur $>$ Menüs}
administriert. Dabei stellt das Menü mit dem Namen "`Main menu"' das Hauptmenü
im Kopfbereich der Seite dar.

Zusätzlich muss das Menü in der Fußleiste erstellt werden, in dem Links wie
das Impressum, Datenschutzerklärung und Kontaktformular abgelegt sind
bzw.\ werden können.  Dazu muss unter oben genanntem Menü auf \emph{Menü
  hinzufügen} geklickt und der Titel so gewählt werden, dass im rechten
Bereich der maschinenlesbare Name \textit{menu-footer-menu} entsteht. Danach
können Links hinzugefügt werden, die dann im rechten unteren Fußbereich der
Website erscheinen.

\subsubsection{Pages erstellen}
Im Releasepaket sind im Ordner \emph{IHR-Pages} Musterdateien enthalten, die
nach Installation von Drupal über \textit{Inhalt $>$ Inhalt hinzufügen $>$
  Basic Page} eingepflegt werden müssen. Sie stellen verschiedene
HTML-Schnipsel zur Verfügung, mit denen die einzelnen Tabs des
Informationsteils aufgebaut werden.

In der Datei \texttt{indexpage.html} ist der Inhalt der bei Projektende
erstellten Startseite enthalten. 

Für alle Seiten ist als Textformat \emph{Full HTML} auszuwählen.

\section{Drupal update}

Der Drupal Core muss regelmäßig an neue Versionen angepasst werden, um
Sicherheitslücken zu schließen. Eine solche Anpassung ist in der
Drupal-Version~7 nur manuell möglich, siehe die Ausführungen in der Datei
\url{UPGRADE.txt} der Core-Distribution.

Im Wesentlichen ist die neue Core-Distribution in einem Verzeichnis
\texttt{newDrupal/} auszurollen, die deutsche Sprachdatei zu übernehmen,
danach die Verzeichnisse \texttt{oldDrupal/sites/},
\texttt{oldDrupal/themes/bootstrap} und \texttt{oldDrupal/themes/IHR} in das
neue Verzeichnis zu kopieren
\begin{quote}\tt
  cp oldDrupal/profiles/standard/translations/drupal-xx.de.po\\\hspace*{2cm} 
  newDrupal/profiles/standard/translations/\\
  cp -R oldDrupal/sites/ newDrupal/\\
  cp -R oldDrupal/themes/bootstrap newDrupal/themes/\\
  cp -R oldDrupal/themes/IHR newDrupal/themes/
\end{quote}
und dann die Produktiv-Version auf das neue Verzeichnis umzustellen. 

Einzelne Module oder die Sprachdatei können durch eine neue Version ersetzt
werden, indem die neue Version an der Stelle der alten ausgerollt wird. Das
Verzeichnis der alten Version sollte vorher gelöscht werden.

Beim Update des Basis-Themes \emph{bootstrap} muss das Unterverzeichnis
\texttt{bootstrap/theme} übernommen werden.
\begin{quote}\tt
  zip bootstrap\_old.zip -r bootstrap\\ 
  mv bootstrap bootstrap.old\\
  unzip bootstrap-(new).zip\\
  cp -R bootstrap.old/theme bootstrap
\end{quote}
Danach das Verzeichnis \texttt{bootstrap.old} löschen, damit es keine
Interferenzen der Themen gibt.  Falls es mit der neuen Version Probleme gibt
(so beim Versuch des Updates auf die Version 3.21 -- HGG, 2018-10-14), kann die
alte Konfiguration aus dem Backup restauriert werden.

\section{Probleme beheben}

Wenn das Layout nicht richtig erscheint, liegt es wahrscheinlich daran, dass
keine CSS-Datei eingebunden wird.  Dazu muss zunächst die Datei
\texttt{style.scss} im Verzeichnis \texttt{code/IHR/css} kompiliert werden,
z.B. mit dem Befehl
\begin{quote}\tt
  sass style.scss:style.css --style compressed
\end{quote}
nachdem man das sass-rubygem installiert hat. Eine kompilierte Version befindet
sich aber auch im Repo-Verzeichnis \texttt{code/IHR/css}.

\section{Anhang: Vorgehen für das Erstellen des Installationspakets}

Das Paket \texttt{drupalDump.zip} wurde wie folgt erstellt: 

\begin{enumerate}\itemsep0pt
\item Standard-Installationspaket von der Drupalwebseite laden. Das ist die
  Grundlage.
\item Mit Hilfe von „Backup \& Migrate“ ein Datenbankbackup der aktuellen
  Installation erstellen. (Nur „Default Database“, nicht „Entire Site“ oder
  ähnliches.)
\item Datenbankbackup in einem Ordner hinterlegen, auf den Drupal dann Zugriff
  hat (hier im Installationspaket: \texttt{sites/default}).
\item Die Module aus der bestehenden Drupalinstallation in das
  Installationspaket übertragen (aus den Ordnern \texttt{modules} und
  \texttt{sites/all/modules}).
\item Die Themes aus der bestehenden Drupalinstallation in das
  Installationspaket übertragen (aus dem Ordner \texttt{themes}).
\item Die Sprachdatei \texttt{drupal-7.39.de.po} im Ordner
  \texttt{profiles/standard/translations} einfügen.
\item Ein Kopie \texttt{settings.php} der Datei \texttt{default.settings.php}
  im Ordner \texttt{sites/default} anlegen.
\end{enumerate}
Dies wird dann alles zu einer zip-Datei zusammengepackt.


\end{document}
